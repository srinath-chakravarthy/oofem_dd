\documentclass[a4paper]{article}
\usepackage{amsmath}
\usepackage{ifluatex}
\ifluatex
 \usepackage{fontspec}
  %\setmainfont[Ligatures=TeX]{xits}
  %\setmathfont[math-style=ISO]{xits-math}
 \setmainfont[Ligatures=TeX]{Latin Modern Roman}
 \fontspec[SmallCapsFeatures={Letters=SmallCaps}]{Latin Modern Roman}
 \usepackage{unicode-math}
 \setmathfont[math-style=ISO]{lmmath-regular.otf}
 %\setmathfont[math-style=ISO]{xits-math.otf}
\else
 \usepackage[utf8]{inputenc}
\fi
\usepackage{microtype}
\usepackage{html}
\usepackage{graphicx}
\usepackage[usenames,dvipsnames,svgnames,table]{xcolor}
\usepackage{listings}   
%\usepackage{tikz}
%\usepackage{hyperref}
\usepackage[top=0.1\paperheight, bottom=0.1\paperheight, left=0.1\paperwidth, right=0.1\paperwidth]{geometry}
\usepackage{tabularx}
\usepackage{ifthen}

\newcommand{\oofem}{\htmladdnormallink{OOFEM}{http://www.oofem.org}\ }
\newcommand{\oofemlnk}[1]{\htmladdnormallink{#1}{http://www.oofem.org}\ }
\newcommand{\bp}{\htmladdnormallink{Bo\v{r}ek Patz\'{a}k}{http://mech.fsv.cvut.cz/~bp/bp.html}}


\renewcommand{\topfraction}{1.99}	% 99% of page top can be a float
\renewcommand{\bottomfraction}{0.99}	% 99% of page bottom can be a float
\renewcommand{\textfraction}{0.01}	% only 1% of page must to be the text
\renewcommand{\floatpagefraction}{0.99} % 99% of whole page can be a float
\setcounter{totalnumber}{5} %maximum floating objects on one page

\newcommand{\mbf}[1]{\boldsymbol{#1}}
\newcommand{\grad}{\vect{\nabla}}
\newcommand{\vect}[1]{\boldsymbol{#1}} % not for Greek letters
\newcommand{\tens}[1]{\boldsymbol{#1}} % A second-order tensor

\newcommand{\strain}{\varepsilon}
\newcommand{\stress}{\sigma}
\newcommand{\dev}{\mathrm{dev}}
\newcommand{\vol}{\mathrm{vol}}

\newcommand{\param}[1]{\texttt{#1}} % Use when referring to attributes in text.
\newcommand{\optional}[1]{[#1]} % Denotes optional parameter
\newcommand{\field}[2]{\param{#1}~\#{\tiny(#2)}} % Attribute #1 = identifier, #2 = type
\newcommand{\optField}[2]{\optional{\field{#1}{#2}}}
\newcommand{\fieldnotype}[1]{\param{#1}}
\newcommand{\optFieldnotype}[1]{[\param{#1}]}
\newcommand{\componentNum}{(\param{num}\#){\tiny(in)}} % For the component number.
\newcommand{\entKeyword}[1]{*\textbf{#1}} % For general keywords, e.g. ``engnmodel''
\newcommand{\entKeywordInst}[1]{\textbf{#1}} % For specific keywords, e.g. ``nlinearstatic''
\newcommand{\entKeywordWithVal}[2]{*\textbf{#1}~\#{\tiny(#2)}} % For general keywords, e.g. ``engnmodel''
\newcommand{\entKeywordInstWithVal}[2]{\textbf{#1}~\#{\tiny(#2)}} % For specific keywords, e.g. ``nlinearstatic''

% macros for items related to parallel
\newcommand{\Pmode}[1]{{\sffamily #1}}
\newcommand{\Pparam}[1]{\param{#1}}
\newcommand{\oofemParallel}[1]{$\langle${#1}$\rangle$}
\newcommand{\PentKeyword}[1]{\oofemParallel{\entKeyword{#1}}}
\newcommand{\PentKeywordInst}[1]{\oofemParallel{\entKeywordInst{#1}}}
\newcommand{\Pfield}[2]{\oofemParallel{\field{#1}{#2}}}
\newcommand{\Pfieldnotype}[1]{\oofemParallel{\fieldnotype{#1}}}
\newcommand{\PoptField}[2]{\oofemParallel{\optField{#1}{#2}}}
\newcommand{\PoptFieldnotype}[1]{\oofemParallel{\optFieldnotype{#1}}}

\begin{htmlonly}
\newenvironment{record}[1][]{\begin{tabular}{|ll}}{\end{tabular}\\}
\newcommand{\recentry}[2]{{#1}&{#2}\\}
\end{htmlonly}
%begin{latexonly}
\newcounter{rcc}
\newenvironment{record}[1][\textwidth]{\setcounter{rcc}{0}\begin{tabular*}{#1}{|ll@{\extracolsep{\fill}}r}}{\end{tabular*}\\}
%\newenvironment{record}{\begin{tabular}{|llr}}{\end{tabular*}\\}
\newcommand{\recentry}[2]{\ifthenelse{\value{rcc}>0}{&$\backslash$ \\}{\setcounter{rcc}{1}}{#1}&{#2}}
%end{latexonly}

\lstdefinelanguage{oofeminput}
{
morekeywords={Node, node, Truss2d}, 
morecomment=[l]{\#},
emph={coords, bc, ic, nodes, mat, crossSect}, 
}

\lstdefinestyle{oofem}{
  belowcaptionskip=1\baselineskip,
  frame=single,
  rulecolor=\color{gray},
  xleftmargin=\parindent,
  basicstyle=\footnotesize,
  keywordstyle=\color{red}\bfseries,
  commentstyle=\itshape, %%\color{green!40!black},
  emphstyle=\color{red},
  captionpos=b,
}

%%\bodytext{
%%bgcolor="#e5f5ff" text="#000000" link="#0000FF" vlink="#FF0000"
%%alink="#FF0000"
%%}

%%\definecolor{oofembgr}{RGB}{229,245,255}
%%\pagecolor{oofembgr}

\begin{document}
%begin{latexonly}
\title{
\oofemlnk{\centerline{\includegraphics[width=0.7\textwidth]{oofem-logo-contour}}}
Input Data Format Specification}
\author{\bp \\ \\
Czech Technical University\\
Faculty of Civil Engineering\\
Department of Structural Mechanics\\
Th\'akurova 7, 166 29 Prague, Czech Republic
}
\maketitle
%end{latexonly}
\begin{htmlonly}
\begin{center}
  \oofemlnk{\centerline{\includegraphics[width=0.7\textwidth]{oofem-logo-contour}}}
{\Large Input Data Format Specification} \\ \\
{\bp \\
Czech Technical University\\
Faculty of Civil Engineering\\
Department of Structural Mechanics\\
Th\'akurova 7, 166 29 Prague, Czech Republic
}
\end{htmlonly}

\newpage
\tableofcontents
\newpage
\section{Introduction}
This manual describes in details the format and structure of OOFEM
text input file. Input file can be prepared in any text editor or can be generated
by a conversion program or FEM pre-processor.

\Pmode{Some parts of this document are relevant only to the parallel enabled version of the code.
These parts are distinguished by typing the text in sans serif font family.}
\Pmode{The parallel model is based on domain decomposition. In this mode a set of input files must be provided, each one corresponding to 
particular process and related domain partition. The corresponding domain partition input file is referred as partition input file. 
The name of partition input file is composed from two parts, the fist part (referred as base
name) is user-defined, the second part (called partition name), divided from first part by
full-stop, is corresponding partition rank number. The partitions are numbered
from zero. All partitions input files must have the
same base name.

The parallel version requires the unique global numbering  for
dof managers (nodes) and elements. The global
numbering is necessary to link partitions together. However, the dof
managers and optionally elements at interpartion boundaries,
should be explicitly  marked, in order to distinguish different type
of relations between their remote counterparts (see further).
}

\section{Running the code}
\label{_running_the_code}
The program can be executed by typing\\
\texttt{oofem [option [parameter]] ...}\\
on the command line prompt with the following command line options:\\[1em]
\begin{tabularx}{\textwidth}{l|X}
\hline
\textbf{\mbox{-v}} & Prints oofem version.\\
\textbf{\mbox{-f~string}} & path to oofem input file name, if not present, program interactively
reads this parameter.\Pmode{When invoking the solver in parallel mode, only base name of input file should be specified, 
the partition name is appended automatically. On the other hand, the partition name is not appended to output file name, as specified
in output file record (see section \ref{_OutputFileRecord}).}\\
\textbf{\mbox{-r~int}} & Restarts the analysis from given solution step. The corresponding
context file (*.osf) must exist.\\
\textbf{\mbox{-rn}} & Turns on the equation renumbering. Default is off.\\
\textbf{\mbox{-ar~int}} & 
Restarts the adaptive computation from given solution step.
Requires the corresponding context file (*.osf) and domain input file
(*.din) to exists. The domain input file describes the new mesh, its
syntax is identical to syntax of input file, but it does not contains
the output file record, job description record and analysis record.\\
\textbf{\mbox{-l~int}} & Sets treshold for log messages (Errors=0, Warnings=1,
Relevant=2, Info=3, Debug=4).\\
\textbf{\mbox{-qo~string}} & Redirect the standard output stream (stdout) to given file.\\
\textbf{\mbox{-qe~string}} & Redirect standard error stream (stderr) to given file.\\
\textbf{\mbox{-c}} & Forces the creation of context file for each solution step.\\
\hline
\end{tabularx}\\[1em]

\Pmode{
The parallel version uses the MPI (Message Passing Interface) standard for message-passing communication. Thus, to execute OOFEM program for parallel (indicated by the \texttt{-p} flag),
users must know the procedure for executing/scheduling MPI jobs on the particular system(s).
For instance, when using the MPICH implementation of MPI and many others, the following command initiates a program that uses eight processors: \\[2mm]
\texttt{mpirun -np 8 oofem -p program\_options}
}




\section{Syntax and general rules}

Input file is composed of records. In the current implementation, each record is represented by one line in input file.
The order of records in file is compulsory, and it has following structure:
\begin{enumerate}
\item output file record, see section \ref{_OutputFileRecord},
\item job description record, see section \ref{_JobDescriptionRecord},
\item analysis record, see section \ref{_AnalysisRecord},
\item domain record, see section \ref{_DomainRecord},
\item output manager record, see section \ref{_OutputManagerRecord},
\item components size record, see section \ref{_ComponentsSizeRecord},
\item node \& element-side  record(s), see section \ref{_NodeElementSideRecords},
\item element record(s), see section \ref{_ElementsRecords},
\item cross section record(s), see section \ref{_CrossSectionRecords},
\item material type  record(s), see section \ref{_MaterialTypeRecords},
\item nonlocal barriers record(s), see section \ref{_NonlocalBarrierRecords},
\item load, boundary  conditions record(s), see section
  \ref{_LoadBoundaryInitialConditions},
\item initial conditions record(s), see section \ref{_InitialConditions},
\item time functions record(s), see section \ref{_TimeFunctionsRecords}.
\item optional xfem manager and associated record(s), see section \ref{_XFEMManagerRecords}
\end{enumerate}
When input line begins with '\#' character, then it is ignored  by the parser and
can serve as a comment inside input file.

The individual records consist of record keyword followed by one or more attributes. Each attribute is identified by
its keyword, which can be followed by attribute value(s). Some attributes have no values.
The order of attributes in the record is optional.

Sometimes, the record keyword itself can be variable, taking on a restricted range of possible values.
As an example, OOFEM has element record, desribing particulat element, and record keyword determines the  particular element type.
In this case, the record keyword is preceded by star. We call such record keyword as entity keyword. The possible
substitutions for entity keyword are typed using \entKeywordInst{Typewriter} font family.
Often, some attributes are specific to particular entity keyword. Then the general format of record is described and entity specific attributes are described separately. The possible attributes are then union of general and entity specific attributes.

\begin{lstlisting}[style=oofem, language=oofeminput, caption={Example of input records}]
# nodal records
Node 1 coords 3 0. 0. 0. bc 2 1 1
Node 2 coords 3 0. 0. 2. bc 2 1 0 
# element record
Truss2d 1 nodes 2 1 2 mat 1 crossSect 1
\end{lstlisting}

Each attribute value has a specific type, which describe its size and layout. To describe the type of an attribute, the following notation is used: 
\field{Keyword}{type}, where type determines the attribute type and \# is the placeholder for the attribute value.
The possible types of attribute values are following:
\begin{itemize}
\item \textbf{in} - integer number.
\begin{lstlisting}[style=oofem, language=oofeminput, moreemph={val1}]
val1 25
\end{lstlisting}

\item \textbf{rn} - real number.
\begin{lstlisting}[style=oofem, language=oofeminput, moreemph={val2}]
val2 -0.234e-3
\end{lstlisting}
\item \textbf{ch} - character (usually for description of unknown type ('d' for
displacement, 't' for temperature, etc.).
\begin{lstlisting}[style=oofem, language=oofeminput, moreemph={val3}]
val3 t
\end{lstlisting}

\item \textbf{ia} - integer array. The format of integer array is
``size~val(1)~...~val(size)'', where size, val(1),...,val(size) are
integer numbers. Values are separated by one or more spaces. As an example, consider the integer array attribute called \param{bc}$=\{1,2,3\}$:
\begin{lstlisting}[style=oofem, language=oofeminput, moreemph={bc}]
bc 3 1 2 3
\end{lstlisting}
\item \textbf{ra} - real array. The format of real array is
``size~val(1)~...~val(size)'', where size is integer number and val(1),
..., val(size) are real numbers. Values are separated by one or more spaces. As an example, consider the real array attribute called \param{coords}$=\{1.0,2.0,3.0\}$:
\begin{lstlisting}[style=oofem, language=oofeminput, moreemph={coords}]
coords 3 1.0 2.0 3.0
\end{lstlisting}
\item \textbf{rm} - real matrix, format of real matrix is\\
``rows~columns~\{val(1,1)~val(1,2)~...;~val(2,1)~...\}'', where ``rows'' and ``columns'' are integer numbers and val(1,1),
..., are real numbers. Columns are seperated by space or comma and lines by semicolon.
As an example, consider the real matrix attribute called $\param{mat1}=\left[\begin{array}{ccc}1.0&-1.0&0.0\\2.0&2.5&5.0\end{array}\right]$:
\begin{lstlisting}[style=oofem, language=oofeminput, moreemph={mat1}]
mat1 2 3 \{1.0 -1.0; 0.0 2.0; 2.5 5.0\}
\end{lstlisting}

\item \textbf{dc} - dictionary. Dictionary consist of pairs, each pair has key
(character type) and its associated value (integer type).
Format of dictionary is
``size~key(1)~val(1)~...~key(size)~val(size)'', where size is integer
number, key(1),...,key(size) are single character values, and val(1),
..., val(size) are real numbers. Values are separated by one or more spaces;
\begin{lstlisting}[style=oofem, language=oofeminput, moreemph={dict1}]
dict1 2 a 1.0 v 0.0 
\end{lstlisting}
\item \textbf{rl} - range list. Range list syntax is \{ number1 .. numberN (start1
end1) (start2 end2)\}. The enclosing brackets are compulsory. The range
list represent list of integer values. Single values can be specified
using single values (number1, .., NumberN). The range of values
(all numbers from startI to endI including startI and endI can be
specified using range value in the form (startI endI). The range is
described using its start and end values enclosed in parenthesis.
Any number of ranges and single values can be used to specify range list.
\begin{lstlisting}[style=oofem, language=oofeminput, moreemph={range1}]
range1 { 1 7 8 (10 20) (25 30) }
\end{lstlisting}
\item \textbf{et} - entity type. For  example, it describes the finite element
type. Possible type values are mentioned in specific sections.
\item \textbf{s} - character string. The string have to be enclosed
in quotes ("") following after corresponding keyword.
\begin{lstlisting}[style=oofem, language=oofeminput, moreemph={string1}]
string1 ``string example''
\end{lstlisting}
\item \textbf{expr} - function expression. The expression have to be enclosed
in quotes (""). The expression is evaluated by internal parser and
represent mathematical expressions as a function of certain variables.
The variable names and meaning are described in specific sections.
The usual arithmetic operators like -,+,*,/ are supported and their
evaluation order is taken into account. The evaluation order can be
changed using parenthesis. Several built-in functions are supported
(sqrt, sin, cos, tan, atan, asin and acos) - these must be typed using
lowercase letters and their arguments must be enclosed in parenthesis.
\begin{lstlisting}[style=oofem, language=oofeminput, moreemph={expr1}]
expr1 ``2.0*sin(t)/3.0''
\end{lstlisting}

\end{itemize}

The general format of record is \\

\noindent
\begin{record}
  \recentry{\field{record\_keyword}{type}}{\optField{attribute1\_keyword}{type}}
  \recentry{}{...}
  \recentry{}{\optField{attributeXX\_keyword}{type}}
\end{record}

The keywords and their values are separated by one or more spaces. Please note, that a single record cooresponds to one input line in input file. 

%In a special case of analysis record, the input record can be followed
%by optional meta-step input records (see section
%\ref{_AnalysisRecord}). Then certain attributes originally in analysis
%record should appear in meta-step record instead. This is marked by
%adding ``M'' superscript to keyword. Then the attribute format is \field{Keyword$^M$}{type}.

When some attribute is enclosed in brackets [~], then it's use is optional
and often overwrites the default behavior or adds additional (but
optional) information or property (for example adds a loading to
node).

\Pmode{If any attribute (and thus its keyword and value) is enclosed within angle brackets $\langle$~$\rangle$ then it is related to parallel version of oofem is not available in sequential version.}

{\flushleft \bf Example of input record.}\\
As an example, consider the following record description:\\
\begin{record}
  \recentry{\entKeywordInstWithVal{Particle}{in}}{\field{color}{in} \field{mass}{rn}}
  \recentry{}{\field{coords}{ra} \field{name}{s}}
\end{record}
The following listing shows the corresponding, properly formatted, input record:
\begin{lstlisting}[style=oofem, language=oofeminput, caption={Corresponding input record}, moreemph={color, mass, name}, morekeywords={Particle}]
Particle 2 color 5 mass 0.18 coords 3 0.0 1.0 2.0 name "P1_36"
\end{lstlisting}



\section{Output file record}
\label{_OutputFileRecord}
This record has no keywords and contains a character string, which
describes the path to output file. If the file with the same name exists,
it will be overwritten.

\section{Job description record}
\label{_JobDescriptionRecord}
This record has no keywords and contains a character string, which
describes the job. This description will appear in the output
file.

\section{Analysis record}
\label{_AnalysisRecord}
This record describes the type of analysis, which should be
performed. The analysis record can be splitted into optional meta-step input records (see below). 
Then certain attributes originally in analysis record can be specified independently for each meta-step. This is marked by
adding ``M'' superscript to keyword. Then the attribute format is \field{Keyword$^M$}{type}.

The general format of this record can be specified in
\begin{itemize}
\item ``standard-syntax''\\
  \begin{record}[0.9\textwidth]
    \recentry{\entKeyword{AnalysisType}}{\field{nsteps}{in}}
    \recentry{}{\optField{renumber}{in}}
    \recentry{}{\optField{profileopt}{in}}
    \recentry{}{\field{attributes}{string}}
    \recentry{}{\optField{ninitmodules}{in}}
    \recentry{}{\optField{nmodules}{in}}
    \recentry{}{\optField{nxfemman}{in}}
  \end{record}
\item ``meta step-syntax''\\
  \begin{record}[0.9\textwidth]
    \recentry{\entKeyword{AnalysisType}}{\field{nmsteps}{in}}
    \recentry{}{\optField{ninitmodules}{in}}
    \recentry{}{\optField{nmodules}{in}}
    \recentry{}{\optField{nxfemman}{in}}
  \end{record}
  immediately followed by \param{nmsteps} meta step records with the following syntax:\\
  \begin{record}[0.9\textwidth]
    \recentry{}{\field{nsteps}{in} \field{attributes}{string}}
  \end{record}
  The \param{nmsteps} parameter determines the number of ``metasteps''. The
meta step represent sequence of solution steps with common attributes.
There is expected to be \param{nmsteps} subsequent metastep records.
The meaning of meta step record parameters (or analysis record
parameters in ``standard syntax'') is following:
\begin{itemize}
\item \param{nsteps} - determines number of subsequent solution steps
within me\-ta\-step.
\item \param{renumber} - Turns out renumbering after each time step.
Necessary when Dirichlet boundary conditions change during simulation.
Can also be turned out by the executeable flag \texttt{-rn}.
\item \param{profileopt} - Nonzero value turns on the
equation renumbering to optimize the profile of characteristic matrix
(uses Sloan algorithm). By default, profile optimization is not
performed. It will not work in parallel mode.
\item \param{attributes} - contains the metastep related attributes of
analysis (and solver), which are valid for corresponding solution
steps within meta step. If used in standard syntax, the attributes are
valid for all solution step.
\item \param{ninitmodules} - number of initialization module records for given
problem. The initialization modules are specified after meta step section (or
after analysis record, if no metasteps are present). Initialization modules
allow to initialize the state variables by values previously computed by 
external software. The available initialization modules are described in section
\ref{InitModulesSec}.
\item \param{nmodules} - number of export module records for given
problem. The export modules are specified after initialization modules. Export modules
allow to export computed data into external software for
postprocessing. The available export modules are described in section
\ref{ExportModulesSec}.
\item \param{nxfemman} - 1 implies that an XFEM manager is created, 0 implies
that no XFEM manager is created. The XFEM manager stores a list of enrichment
items. The syntax of the XFEM manager record and related records is described in
section \ref{_XFEMManagerRecords}.
\item \param{eetype} - optional error estimator type used for the problem.
Used for adaptive analysis, but can also be used to compute and write error estimates to the output files. See adaptive engineering models for details.
\end{itemize}

\end{itemize}
Not all of analysis types support the metastep syntax, and if
not mentioned, the standard-syntax is expected.
Currently, supported analysis types are

\begin{itemize}
\item Linear static analysis, see section \ref{LinearStatic},
\item Eigen value dynamic, see section \ref{EigenValueDynamic},
\item Direct explicit  nonlinear dynamics, see section
\ref{NlDEIDynamic},
\item Direct explicit (linear) dynamics, see section \ref{DEIDynamic},
\item Implicit linear dynamic, see section \ref{DIIDynamic},
\item Incremental \textbf{linear} static problem, see section \ref{IncrementalLinearStatic},
\item Non-linear static analysis, see section \ref{NonLinearStatic}.
\end{itemize}

\subsection{Linear static analysis}
\label{LinearStatic}
\begin{record}
  \recentry{\entKeywordInst{LinearStatic}}{\field{nsteps}{in}}
  \recentry{}{\optField{sparselinsolverparams}{...}}
  \recentry{}{\optField{sparselinsolverparams}{...}}
\end{record}

Linear static analysis.
Parameter \param{nsteps} indicates the number of loading cases.
Series of loading cases is maintained as sequence of time-steps.
For each load case an auxiliary time-step is generated with time
equal to load case number.
Load vectors for each load case are formed as load vectors at
this auxiliary time.

The  \param{sparselinsolverparams} parameter describes the sparse
linear solver attributes and is explained in section \ref{sparselinsolver}.
\Pmode{Can be used in parallel mode. The input record must be the same
for all processors. At present, parallel version requires PETSc module.}


\subsection{LinearStability}
\label{LinearStability}
\begin{record}
  \recentry{\entKeywordInst{LinearStability}}{\field{nroot}{in}}
  \recentry{}{\field{rtolv}{rn}}
  \recentry{}{\optField{eigensolverparams}{...}}
\end{record}
Solves linear stability problem. Only first \param{nroot} smallest
eigenvalues and corresponding eigenvectors will be computed.
Relative convergence tolerance is specified using \param{rtolv} parameter.

The \param{eigensolverparams} parameter describes the sparse
linear solver attributes and is explained in section \ref{eigensolverssection}.
\Pmode{Can be used in parallel mode. The input record must be the same
for all processors. Parallel version requires PETSc and SLEPc modules.}

\subsection{EigenValueDynamic}
\label{EigenValueDynamic}
\begin{record}
  \recentry{\entKeywordInst{EigenValueDynamic}}{\field{nroot}{in}}
  \recentry{}{\field{rtolv}{rn}}
  \recentry{}{\optField{eigensolverparams}{...}}
\end{record}

Represents the eigen value dynamic analysis.
Only \param{nroot} smallest eigenvalues and corresponding
eigenvectors will be computed. Relative convergence criteria is
governed using \param{rtolv} parameter.

The \param{eigensolverparams} parameter describes the sparse
linear solver attributes and is explained in section \ref{eigensolverssection}.
\Pmode{Can be used in parallel mode. The input record must be the same
for all processors. Parallel version requires PETSc and SLEPc modules.}


\subsection{NlDEIDynamic}
\label{NlDEIDynamic}
\begin{record}
  \recentry{\entKeywordInst{NlDEIDynamic}}{\field{nsteps}{in}}
  \recentry{}{\field{dumpcoef}{rn}}
  \recentry{}{\optField{deltaT}{rn}}
\end{record}

Represents the direct explicit  nonlinear dynamic  integration.
The central difference method with diagonal mass matrix is used,
damping matrix is assumed to be proportional to mass matrix, $\mbf{C}
= \mathrm{dumpcoef} * \mbf{M}$, where
$\mbf{M}$ is diagonal mass matrix. \param{deltaT} is time step length used for
integration, which may be reduced by program in order to satisfy
solution stability conditions. Parameter \param{nsteps} specifies
how many time steps will be analyzed.

The parallel version has the following additional syntax:\\ \\
\begin{record}
  \recentry{}{\PentKeyword{commode}}
  \recentry{}{\PoptFieldnotype{nonlocalext}}
\end{record}

\noindent
\Pmode{
The \PentKeyword{commode} keyword can be one from following:
\begin{itemize}
\item \entKeywordInst{nodecutmode} - indicates the node cut partitioning scheme is
used. The cut is led along element edges or surfaces (see
figures \ref{nodecut}, \ref{nodecut-lm}, and \ref{nodecut-nlm}).
\item \entKeywordInst{elementcutmode} - the element cut partitioning is used. The
partitioning cut led across the element edges or surfaces (see
figures \ref{elmentcut} and \ref{elmentcut-lm}).
\end{itemize}
\Pmode{
The \param{nonlocalext} turns on the nonlocal constitutive
extension. The extension considers a band of remote elements involved
in computation of nonlocal variables (see fig. \ref{nodecut-nlm} illustrating
this approach for node-cut partitioning).
}
}

\subsection{DEIDynamic}
\label{DEIDynamic}

\begin{record}
  \recentry{\entKeywordInst{DEIDynamic}}{\field{nsteps}{in}}
  \recentry{}{\field{dumpcoef}{rn}}
  \recentry{}{\optField{deltaT}{rn}}
\end{record}

Represent the \textbf{linear} explicit
integration scheme for dynamic problem solution. The central difference method with diagonal mass matrix is used,
damping matrix is assumed to be proportional to mass matrix, $\mbf{C} = \mathrm{dumpcoef} * \mbf{M}$, where
$\mbf{M}$ is diagonal mass matrix. \param{deltaT} is time step length used for
integration, which may be reduced by program in order to satisfy
solution stability conditions. Parameter \param{nsteps} specifies
how many time steps will be analyzed.

\subsection{DIIDynamic}
\label{DIIDynamic}

\begin{record}
  \recentry{\entKeywordInst{DIIDynamic}}{\field{nsteps}{in}}
  \recentry{}{\field{deltaT}{rn}}
  \recentry{}{\field{alpha}{rn}}
  \recentry{}{\field{beta}{rn}}
  \recentry{}{\field{Psi}{rn}}
\end{record}

Represents direct implicit integration of linear dynamic
problems. Damping is modeled as Rayleigh damping $(\mbf{c} =
\rm{alpha}*\mbf{M} + \rm{beta} * \mbf{K})$. Parameter \param{Psi} determines
integration method used, for\param{Psi} = 1 the Newmark and for \param{Psi} $\ge$ 1.37 the Wilson method
will be used. Parameter \param{deltaT} is required time integration
step length.

\subsection{IncrementalLinearStatic}
\label{IncrementalLinearStatic}

\begin{record}
  \recentry{\entKeywordInst{IncrLinearStatic}}{\field{endOfTimeOfInterest}{rn}}
  \recentry{}{\field{pre\-scri\-bed\-Ti\-mes}{ra}}
\end{record}


Represents incremental \textbf{linear} static problem.
The problem is solved as series of linear solutions and is intended to
be used for solving linear creep problems or incremental perfect plasticity.

Supports the changes of static scheme (applying, removing and changing  boundary conditions)
during the analysis.

Response is computed in times defined by \param{pre\-scri\-bed\-Ti\-mes}
array. These times should include times, when generally the  boundary
conditions are changing, and in other times of interest. (For linear creep
analysis, the values should be uniformly distributed on log-time scale, if no change in
loading or boundary conditions). The time at the end of interested is
specified using \param{endOfTimeOfInterest} parameter.

\subsection{NonLinearStatic}
\label{NonLinearStatic}
\entKeywordInst{NonLinearStatic} \\
Non-linear static analysis. The problem can be solved under
direct load or displacement control, indirect control, or by
their arbitrary combination.
\Pmode{Can be used in parallel mode. The input record must be the same
for all processors. At present, parallel version requires PETSc module.}
By default all
material nonlinearities will be included, geometrical not.
To include geometrically nonlinear effect one must specify
level of non-linearity in element records.
There are two different ways, how to specify the parameters - the
extended and standard syntax.
\subsubsection{Extended syntax}
The extended syntax uses the ``metastep'' concept and has the
following format:
%%\begin{enumerate}
%%\item[0]

\noindent
\begin{record}
  \recentry{\entKeywordInst{NonLinearStatic}}{\optField{nmsteps}{in}}
  \recentry{}{\field{nsteps}{in}}
  \recentry{}{\optField{contextOutputStep}{in}}
  \recentry{}{\optField{sparselinsolverparams}{string}}
  \recentry{}{\optField{nonlinform}{in}}
  \recentry{}{\PoptField{nonlocstiff}{in}}
  \recentry{}{\PoptFieldnotype{nonlocalext}}
  \recentry{}{\PoptFieldnotype{loadbalancing}}
\end{record}
This record is immediately followed by metastep records with the format described below.
The analysis parameters have following meaning
\begin{itemize}
\item \param{nmsteps} - determines the number of
``metasteps'', default is 1.
\item \param{nsteps} - determines number of solution steps.
\item \param{contextOutputStep} - causes the context file to be
created for every con\-text\-Out\-put\-Step-th step and when needed. Useful for
postprocessing.
\item The  \param{sparselinsolverparams} parameter describes the sparse
linear solver attributes and is explained in section \ref{sparselinsolver}.
\item \param{nonlinform} - formulation of non-linear problem. If == 1 (default), total Lagrangian formulation in undeformed original shape is used (first-order theory). If == 2, the equlibrated displacements are added to original ones and updated in each time step (second-order theory).
\item \Pmode{\param{nonlocstiff} - determines whether the tangent stiffness
extension for nonlocal models is activated. If == 0 (default) this
option is not active. If == 1 the support for nonlocal tangent
stiffness is activated.}
\item \Pmode{The \param{nonlocalext} turns on the nonlocal constitutive
extension. The extension considers a band of remote elements involved
in computation of nonlocal variables (see fig. \ref{nodecut-nlm} illustrating
this approach for node-cut partitioning)}.
\item \Pmode{The  \param{loadbalancing} parameter describes the dynamic load balancing
attributes and is explained in section \ref{dynamicloadbalancing}}.
\end{itemize}

%The metasteps \componentNum\ is the metastep number and the
%\param{nsteps} - determines number of solution steps for corresponding
%metastep.
\noindent
The metastep record has the following general syntax:\\
\begin{record}
  \recentry{\hspace{10mm}}{\field{nsteps}{in}}
  \recentry{}{\optField{controlmode}{in}}
  \recentry{}{\optField{deltat}{rn}}
  \recentry{}{\optField{stiffmode}{in}}
  \recentry{}{\optField{refloadmode}{in}}
  \recentry{}{\field{solverParams}{}}
  \recentry{}{\optField{sparselinsolverparams}{string}}
  \recentry{}{\optField{donotfixload}{}}
\end{record}

where
\begin{itemize}
\item[-] \param{controlmode} - determines the type of solution
control used for corresponding meta step. if == 0 then indirect
control will be used to control solution process (arc-length method, default). if == 1
then direct displacement or load control will be used (Newton-Raphson solver). In the later
mode, one can apply the prescribed load increments as well as
control displacements.
\item[-] \param{deltaT} - is time step length. If not specified, it is
set equal to 1,0. Each solution step has associated the corresponding
intrinsic time, at which the loading is generated. The \param{deltaT}
determines the spacing between solution steps on time scale.
\item[-] \param{stiffMode} - If == 0 (default) then tangent stiffness will be used
at new step beginning and whenever numerical method will ask for
stiffness update. If == 1 the use of secant tangent will be forced.
The secant stiffness will be used at new step beginning
and whenever numerical method will ask for stiffness update.
If == 2 then original elastic stiffness will
be used during the whole solution process.
\item[-] The \param{refloadmode} parameter determines how the reference force load vector
is obtained from given totalLoadVector and initialLoadVector.
The initialLoadVector describes the part of loading which does not scale. Works only for force loading, other non-force components (temperature, prescribed displacements should always given in total values).
If \param{refloadmode} is 0 (rlm\_total, default) then the reference incremental load vector is defined as
totalLoadVector assembled at given time.
If \param{refloadmode} is 1 (rlm\_inceremental) then the reference load vector is
obtained as incremental load vector at given time.
\item[-] \param{solverParams} - parameters of solver. The solver type
is determined using \param{controlmode}.
\item[-] The  \param{sparselinsolverparams} parameter describes the sparse
linear solver attributes and is explained in section
\ref{sparselinsolver}.
\item[-] By default, reached load at the end of
metastep will be maintained in subsequent steps as fixed, non scaling
load and load level will be reset to zero. This can be changed using keyword \param{donotfixload}, which if
present, causes the loading to continue, not resetting the load
level. For the indirect control the reached loading will not be
fixed, however, the new reference loading vector will be assembled for
the new metastep.
\end{itemize}

The direct control corresponds to \param{controlmode}=1 and the Newton-Raphson solver is used.
Under the direct control, the total load
vector assembled for specific solution step represents the load level,
where equilibrium is searched. The implementation supports also
displacement control -  it is possible to prescribe one or more
displacements by applying ``quasi prescribed'' boundary
condition(s)\footnote{Hovewer, the problem does not support the
changes of static system. But it is possible to apply direct
displacement control without requiring BC applied (see nrsolver
documentation). Therefore it is possible to combine
direct displacement control with direct load control or indirect
control.}
The load level then represents the
time, where the equilibrium has been found. The Newton-Raphson solver parameters (\param{solverParams}) for
load-control are:\\
\begin{record}
  \recentry{\hspace{10mm}}{\field{maxiter}{in}}
  \recentry{}{\optField{minsteplength}{in}}
  \recentry{}{\optField{minIter}{in}}
  \recentry{}{\optField{manrmsteps}{in}}
  \recentry{}{\optField{ddm}{ia} \optField{ddv}{ra} \optField{ddltf}{in}}
  \recentry{}{\optField{linesearch}{in} \optField{lsearchamp}{rn}}
  \recentry{}{\optField{lsearchmaxeta}{rn} \optField{lsearchtol}{rn}}
  \recentry{}{\optional{\field{nccdg}{in} \field{ccdg1}{ia} ... \field{ccdgN}{ia}  }}
  \recentry{}{\field{rtolv}{rn} \optField{rtolf}{rn} \optField{rtold}{tn}}
  \recentry{}{\optField{initialGuess}{rn}}
\end{record}
where
\begin{itemize}
\item \param{maxiter} determines the maximum
number of iterations allowed to reach equilibrium. If equilibrium is
not reached, the step length (corresponding to time) is reduced.
\item \param{minsteplength} parameter is the minimum step length
allowed.
\item \param{minIter} - minimum number of iterations which always proceed during the iterative solution.
\item If \param{manrmsteps} parameter is nonzero, then the modified
N-R scheme is used, with the stiffness updated after
\param{manrmsteps} steps.
\item \param{ddm} is array specifying the degrees of freedom,
which displacements are controlled.
Let the number of these DOFs is N.
The format of \param{ddm} array is 2*N dofman1 idof1
dofman2 idof2 ... dofmanN idofN, where the dofmani is the number of i-th dof manager  and idofi is the
corresponding DOF number.
\item \param{ddv} is array of relative weights of controlled
displacements, the size should be equal to N. The actual value of
prescribed dofs is defined as a product of its weight and the value of
load time function specified using \param{ddltf} parameter (see
below).
\item \param{ddltf} number of load time function, which is used to
evaluate the actual displacements of controlled dofs.
\item \param {linesearch} nonzero value turns on line search
  algorithm. The \param{lsearchtol} defines tolerance (default value
  is 0.8), amplification
  factor can be specified using \param{lsearchamp} parameter (should
  be in interval $(1,10)$), and parameter \param{lsearchmaxeta}
  defines maximum limit on the length of iterative step (allowed range
  is $(1.5,15)$).
\item \param{nccdg} allows to define one or more DOF groups, that are used for evaluation of convergence criteria. Each DOF is checked if it is a member of particular group and in this case its contribution is taken into account when evaluating the convergence criteria for that group. By default, if \param{nccdg} is not specified, one group containing all DOF types is created. The value of \param{nccdg} parameter defines the number of DOF type groups. For each group, the corresponding DOF types need to be specified using \param{ccdg\#} parameter, where '\#' should be replaced by group number (numbering starts from 1). This array contains the DofIDItem values, that identify the physical meaning of DOFs in the group. The values and their physical meaning is defined by DofIDItem enum type (see src/oofemlib/dofiditem.h for reference).
\item \param{rtolv} determines relative convergence norm (both for displacement
iterative change vector and for residual unbalanced force vector). Optionally, the \param{rtolf} and \param{rtold} parameters can be used to define
independent relative convergence crteria for unbalanced forces and displacement
iterative change. If the default convergence criteria is used,
the parameters \param{rtolv},\param{rtolf}, and \param{rtold} are real values. If the convergence criteria DOF groups are used (see bellow the description of \param{nccdg} parameter) then they should be specified as real valued arrays of \param{nccdg} size, and individual values define relative convergence criteria for each individual dof group.
\item \param{initialGuess} is an optional parameter with default vaue 0, for which the first iteration of each step starts from the previously converged state and applies the prescribed displacement increments. This can lead to very high strains in elements connected to the nodes with changing prescribed displacements and the state can be far from equilibrium, which may results into slow convergence and strain localization near the boundary. If \param{initialGuess} is set to 1, the contribution of the prescribed displacement increments to the internal nodal forces is linearized and moved to the right-hand side, which often results into an initial solution closer to equilibrium. For instance, if the step is actually elastic, equilibrium is fully restored after the second iteration, while the default method may require more iterations.  
\end{itemize}

The indirect solver corresponds to \param{controlmode}=0 and the CALM
solver is used. The value of reference load vector is determined by
\param{refloadmode} parameter mentioned above at the first step of
each metastep. {\em However, the user must ensure that the same value of
reference load vector could be obtained for all solution steps of
particular metastep (this is necessary for restart and adaptivity to work)}.
The corresponding meta step solver parameters (\param{solverParams}) are:\\
\begin{record}
  \recentry{\hspace{10mm}}{\field{Psi}{rn}}
  \recentry{}{\field{MaxIter}{in}}
  \recentry{}{\field{stepLength}{rn}}
  \recentry{}{\optField{min\-Step\-Le\-ngth}{in}}
  \recentry{}{\optField{initialStepLength}{rn}}
  \recentry{}{\optField{forcedInitialStepLength}{rn}}
  \recentry{}{\optField{reqIterations}{in}}
  \recentry{}{\optField{maxrestarts}{in}}
  \recentry{}{\optField{minIter}{in}}
  \recentry{}{\optField{manrmsteps}{in}}
  \recentry{}{\optField{hpcmode}{in} \optField{hpc}{ia} \optField{hpcw}{ra}}
  \recentry{}{\optField{linesearch}{in} \optField{lsearchamp}{rn}}
  \recentry{}{\optField{lsearchmaxeta}{rn} \optField{lsearchtol}{rn}}
  \recentry{}{\optional{\field{nccdg}{in} \field{ccdg1}{ia} ... \field{ccdgN}{ia}}}
  \recentry{}{\field{rtolv}{rn} \optField{rtolf}{rn} \optField{rtold}{rn}}
  \recentry{}{\optField{pert}{ia} \optField{pertw}{ra}}
  \recentry{}{\optField{rpa}{rn} \optField{rseed}{in}}
\end{record}
where
\begin{itemize}
\item \param{Psi} - CALM $\Psi$ control parameter. For $\Psi$ = 0
displacement control is applied. For nonzero values the load control
applies together with displacement control (ALM). For large $\Psi$
load control apply.
\item \param{MaxIter} - determines the maximum number of iteration allowed to
reach equilibrium state. If this limit is reached, restart follows
with smaller step length.
\item \param{stepLength} - determines the maximum value of arc-length (step length).
\item \param{min\-Step\-Le\-ngth} - minimum step length. The step length will never be
smaller. If convergence problems are encountered and step length cannot
be decreased, computation terminates.
\item \param{initialsteplength} - determines the initial step length (the arc-length). If not provided, the maximum step length (determined by \param{stepLength} parameter) will be used as the value of initial step length. 
\item \param{forcedInitialStepLength} - When simulation is restarted, the last predicted step length is used. Use \param{forcedInitialStepLength} parameter to override the value of step length. This parameter will also override the value of initial step length set by \param{initialsteplength} parameter.
\item \param{reqIterations} - approximate number of iterations controlled by changing the step length.
\item \param{maxrestarts} - maximum number of restarting computation when convergence not reached up to \param{MaxIter}.
\item \param{minIter} - minimum number of iterations which always proceed during the iterative solution. \param{reqIterations} are set to be the same, \param{MaxIter} are increased if lower.
\item \param{manrmsteps} - Forces the use of accelerated
Newton Raphson met\-hod, where stiffness is updated after
\param{manrmsteps} steps. By default, the modified NR
method is used (no stiffness update).
\item \param{hpcmode} Parameter determining the alm mode.
Possible values are: 0 - (default) full ALM with quadratic constrain and
all dofs, 1 - (default, if \param{hpc} parameter used) full ALM with quadratic constrain, taking into account
only selected dofs (see \param{hpc} param), 2 - linearized constrain
in displacements only, taking into account only selected dofs with
given weight (see \param{hpc} and \param{hpcw} parameters).
\item \param{hpc} - Special parameter for Hyper-plane control, when only
selected DOFs are taken account in ALM step length
condition. Important mainly for material nonlinear problems with
strong localization. This array selects the degrees of freedom,
which displacements are controlled. Let the number of these DOFs be N.
The format of \param{ddm} array is 2*N dofman1 idof1
dofman2 idof2 ... dofmanN idofN, where the dofmani is the number of i-th dof manager  and idofi is the
corresponding DOF number.
\item \param{hpcw} Array of DOF weights in linear constraint. The
dof ordering is determined by \param{hpc} parameter, the size of the array should
be N.
\item \param{linesearch} nonzero value turns on line search
  algorithm. The \param{lsearchtol} defines tolerance, amplification
  factor can be specified using \param{lsearchamp} parameter (should
  be in interval $(1,10)$), and parameter \param{lsearchmaxeta}
  defines maximum limit on the length of iterative step (allowed range
  is $(1.5,15)$).
\item \param{nccdg} allows to define one or more DOF groups, that are used for evaluation of convergence criteria. Each DOF is checked if it is a member of particular group and in this case its contribution is taken into account when evaluating the convergence criteria for that group. By default, if \param{nccdg} is not specified, one group containing all DOF types is created. The value of \param{nccdg} parameter defines the number of DOF type groups. For each group, the corresponding DOF types need to be specified using \param{ccdg\#} parameter, where '\#' should be replaced by group number (numbering starts from 1). This array contains the DofIDItem values, that identify the physical meaning of DOFs in the group. The values and their physical meaning is defined by DofIDItem enum type (see src/oofemlib/dofiditem.h for reference).
\item \param{rtolv} determines relative convergence norm (both for displacement
iterative change vector and for residual unbalanced force vector). Optionally, the \param{rtolf} and \param{rtold} parameters can be used to define
independent relative convergence crteria for unbalanced forces and displacement
iterative change. If the default convergence criteria is used,
the parameters \param{rtolv},\param{rtolf}, and \param{rtold} are real values. If the convergence criteria DOF groups are used (see bellow the description of \param{nccdg} parameter) then they should be specified as real valued arrays of \param{nccdg} size, and individual values define relative convergence criteria for each individual dof group.
\item \param{pert} Array specifying DOFs that should be perturbed after the first iteration of each step. 
Let the number of these DOFs be M.
The format of \param{ddm} array is 2*M dofman1 idof1
dofman2 idof2 ... dofmanN idofN, where the dofmani is the number of i-th dof manager  and idofi is the
corresponding DOF number.
\item \param{pertw} Array of DOF perturbations. The
dof ordering is determined by \param{pert} parameter, the size of the array should
be M.
\item \param{rpa} Amplitude of random perturbation that is applied to each DOF.
\item \param{rseed} Seed for the random generator that generates random perturbations.
\end{itemize}

\subsubsection{Standard syntax}
In this case, all parameters (for analysis as well as for the solver)
are supplied in analysis record. The default meta step is created for
all solution steps required. Then the meta step attributes are
specified within analysis record. The format of analysis record is
then following\\

\noindent
\begin{record}
  \recentry{\entKeywordInst{NonLinearStatic}}{\field{nsteps}{in}}
  \recentry{}{\optField{nonlocstiff}{in}}
  \recentry{}{\optField{contextOutputStep}{in}}
  \recentry{}{\optField{controlmode}{in}}
  \recentry{}{\optField{deltat}{rn}}
  \recentry{}{\field{rtolv}{rn}}
  \recentry{}{\optField{stiffmode}{in}}
  \recentry{}{\field{lstype}{in}}
  \recentry{}{\field{smtype}{in}}
  \recentry{}{\field{solverParams}{}}
  \recentry{}{\optField{nonlinform}{in}}
  \recentry{}{\PoptField{nonlocstiff}{in}}
  \recentry{}{\PoptFieldnotype{nonlocalext}}
  \recentry{}{\PoptFieldnotype{loadbalancing}}
\end{record}
The meaning of parameters is the same as for extended syntax.

%
\subsection{Adaptive linear static}
\label{AdaptiveLinearStatic}
\begin{record}
  \recentry{\entKeywordInst{Adaptlinearstatic}}{\field{nsteps}{in}}
  \recentry{}{\optField{sparselinsolverparams}{...}}
  \recentry{}{\optField{meshpackage}{in}}
  \recentry{}{\field{errorestimatorparams}{...}}
\end{record}
Adaptive linear static analysis. Multiple loading cases are not
supported. Due to linearity of a problem, the complete reanalysis from
the beginning is done after adaptive remeshing.
After first step the error is estimated, information about required density is generated
(using mesher interface) and solution terminates. If the error
criteria is not satisfied, then the new mesh
and corresponding input file is generated and new analysis should be
performed until the error is acceptable.
Currently, the available error estimator for linear problems is
Zienkiewicz-Zhu. Please note, that adaptive framework requires
specific functionality provided by elements and material models. For
details, see element and material model manuals.
\begin{itemize}
\item[-]
Parameter \param{nsteps} indicates the number of loading cases.
Should be set to 1.
\item[-]
The  \param{sparselinsolverparams} parameter describes the sparse
linear solver attributes and is explained in section \ref{sparselinsolver}.
\item[-]
The \param{meshpackage} parameter selects the mesh package interface,
which is used to generate information about required mesh density for
new remeshing. The supported interfaces are explained in section
\ref{meshpackages}. By default, the T3d interface is used.
\item[-]
The \param{errorerestimatorparams} parameter contains the parameters
of Zien\-kie\-wicz Zhu Error Estimator. These are described in section \ref{errorestimators}.
\end{itemize}


\subsection{Adaptive nonlinear static}

\begin{record}
  \recentry{\entKeywordInst{Adaptnlinearstatic}}{\field{Nonlinearstaticparams}{}}
  \recentry{}{\optField{equilmc}{in}}
  \recentry{}{\optField{meshpackage}{in}}
  \recentry{}{\optField{eetype}{in}}
  \recentry{}{\field{errorestimatorparams}{...}}
\end{record}

Represents Adaptive Non-LinearStatic problem. Solution is performed  as a series of increments (loading or displacement).
The error is estimated at the end of each load increment (after
equilibrium is reached), and based on reached error, the computation
continues, or the new mesh densities are generated and solution
stops. Then the new discretization should be generated.
The truly adaptive approach is supported, so the computation can be
restarted from the last step (see section \ref{_running_the_code}), solution is mapped to new mesh (separate
solution step) and new load increment is applied. Of course, one can start the analysis from
the very beginning using new mesh. Currently, the available
estimators/indicators include only linear
Zienkiewicz-Zhu estimator and scalar error indicator. Please note, that adaptive framework requires
specific functionality provided by elements and material models. For
details, see element and material model manuals.
\begin{itemize}
\item[-]
Set of parameters \param{Nonlinearstaticparams} are related to
nonlinear analysis. They are described in section
\ref{NonLinearStatic}.
\item[-]
Parameter \param{equilmc} determines, whether after mapping of primary
and internal variables to new mesh the equilibrium is restored or not
before new load increment is applied. The possible values are: 0
(default), when no equilibrium is restored, and 1 forcing the
equilibrium to be restored before applying new step.
\item[-]
The \param{meshpackage} parameter selects the mesh package interface,
which is used to generate information about required mesh density for
new remeshing. The supported interfaces are explained in section
\ref{meshpackages}. By default, the T3d interface is used.
\item[-]
Parameter \param{eetype} determines the type of error
estimator/indicator to be used. The parameters
\param{errorestimatorparams} represent set of parameters corresponding
to selected error estimator. For description, follow to section
\ref{errorestimators}.
\end{itemize}
%

\subsection{Stationary transport problem}
\label{StationaryTransport}

\begin{record}
  \recentry{\entKeywordInst{StationaryProblem}}{\field{nsteps}{in}}
  \recentry{}{\optField{sparselinsolverparams}{...}}
  \recentry{}{\optField{exportfields}{ia}}
\end{record}

Stationary transport problem.
Series of loading cases is maintained as sequence of time-steps.
For each load case an auxiliary time-step is generated with time
equal to load case number.
Load vectors for each load case are formed as load vectors at
this auxiliary time.
The  \param{sparselinsolverparams} parameter describes the sparse
linear solver attributes and is explained in section \ref{sparselinsolver}.

If the present problem is used within the context
of staggered-like analysis, the temperature field obtained by the
solution can be exported and made available to any subsequent
analyses. For example, temperature field obtained by present analysis
can be taken into account in subsequent mechanical analysis.
To allow this, the temperature must be ``exported''. This can be done
by adding array \param{exportfields}. This array contains the field
identifiers, which tell the problem to register its primary unknowns
under given identifiers. See file {\it field.h}.
Then the subsequent analyses can get access to exported fields 
and take them into account, if they support such feature.

\subsection{Transient transport problem - linear case}
\label{LinearTransientTransport}

\begin{record}
  \recentry{\entKeywordInst{NonStationaryProblem}}{\field{nsteps}{in}}
  \recentry{}{\field{deltaT}{rn} $|$ \field{deltaTfunction}{in}}
  \recentry{}{\field{alpha}{rn}}
  \recentry{}{\optField{initT}{rn}}
  \recentry{}{\optFieldnotype{lumpedcapa}}
  \recentry{}{\optField{sparselinsolverparams}{..}}
  \recentry{}{\optField{exportfields}{ia}}
  \recentry{}{\optFieldnotype{changingProblemSize}}
\end{record}

\textbf{Linear} implicit
integration scheme for transient transport problems. The generalized
midpoint rule (sometimes called $\alpha$-method) is used for time discretization, with alpha parameter,
which has limits $0\le\alpha\le1$. For $\alpha=0$ explicit Euler
forward method is obtained, for $\alpha=0.5$ implicit trapezoidal rule
is recovered, which is unconditionally stable, second-order accurate
in $\Delta t$, and $\alpha=1.0$ yields implicit Euler backward method,
which is unconditionally stable, and first-order accurate
in $\Delta t$. \param{deltaT} is time step length used for
integration, \param{nsteps} parameter specifies
number of time steps to be solved. It is possible to define \param{deltaTfunction}
with a number referring to corresponding time function, see section~\ref{_TimeFunctionsRecords}.
Variable time step is advantageous when calculating large time intervals. It is strongly suggested
to use nonlinear transport solver due to stability reasons, see section~\ref{TransientTransport}.

The \param{initT} sets the initial time for integration, 0 by default.
If \param{lumpedcapa} is set, then the stabilization of numerical
algorithm using lumped capacity matrix will be used, reducing the
initial oscillations.
See section \ref{StationaryTransport} for an explanation on
\param{exportfields}.

This linear transport problem supports changes in number of equations. It is possible to impose/remove
Dirichlet boundary conditions during solution. This feature is enabled with \param{changingProblemSize}, which ensures
storing solution values on nodes (DoFs) directly. If the problem does not grow/decrease during solution, it is more efficient
to use conventional solution strategy and the parameter should not be mentioned.

Note: This problem type \textbf{requires transport module} and it
can be used only when this module is included in your oofem
configuration.

\subsection{Transient transport problem - nonlinear case}
\label{TransientTransport}

\begin{record}
  \recentry{\entKeywordInst{NlTransientTransportProblem}}{\field{nsteps}{in}}
  \recentry{}{\field{deltaT}{rn} $|$  \field{deltaTfunction}{in}}
  \recentry{}{\field{alpha}{rn}}
  \recentry{}{\optField{initT}{rn}}
  \recentry{}{\optField{lumpedcapa}{}}
  \recentry{}{\optField{nsmax}{in}}
  \recentry{}{\field{rtol}{rn}}
  \recentry{}{\optField{manrmsteps}{in}}
  \recentry{}{\optField{sparselinsolverparams}{...}}
  \recentry{}{\optField{exportfields}{ia}}
  \recentry{}{\optFieldnotype{changingProblemSize}}
\end{record}

Implicit integration scheme for transient transport problems. The generalized
midpoint rule (sometimes called $\alpha$-method) is used for time discretization, with alpha parameter,
which has limits $0\le\alpha\le1$. For $\alpha=0$ explicit Euler
forward method is obtained, for $\alpha=0.5$ implicit trapezoidal rule
is recovered, which is unconditionally stable, second-order accurate
in $\Delta t$, and $\alpha=1.0$ yields implicit Euler backward method,
which is unconditionally stable, and first-order accurate
in $\Delta t$. See matlibmanual.pdf for solution algorithm.

\param{deltaT} is time step length used for integration, \param{nsteps} parameter specifies
number of time steps to be solved. For \param{deltaTfunction} and \param{initT} see section~\ref{LinearTransientTransport}.
Parameter \param{maxiter} determines the maximum
number of iterations allowed to reach equilibrium (default is 30). Norms of
residual physical quantity (heat, mass) described by solution vector and 
the change of solution vector are determined in each iteration.
The convergence is reached, when the norms are less than the value given by \param{rtol}.
If \param{manrmsteps} parameter is nonzero, then the modified
N-R scheme is used, with the left-hand side matrix updated after
\param{manrmsteps} steps.
\param{nsmax} maximum number of iterations per time step, default is 30.
If \param{lumpedcapa} is set, then the stabilization of numerical
algorithm using lumped capacity matrix will be used, reducing the
initial oscillations.

See the Section~\ref{StationaryTransport} for an explanation on
\param{exportfields}. The meaning 
of \param{changingProblemSize} is given in Section~\ref{LinearTransientTransport}.

Note: This problem type \textbf{requires transport module} and it
can be used only when this module is included in your oofem
configuration.

\subsection{Transient incompressible flow - CBS Algorithm}
\label{cbsIncomp}
\begin{record}
  \recentry{\entKeywordInst{CBS}}{\field{nsteps}{in}}
  \recentry{}{\field{deltaT}{}}
  \recentry{}{\optField{theta1}{in}}
  \recentry{}{\optField{theta2}{in}}
  \recentry{}{\optField{cmflag}{in}}
  \recentry{}{[\field{scaleflag}{in} \field{lscale}{in} \field{uscale}{in} \field{dscale}{in}]}
  \recentry{}{\optField{lstype}{in} \optField{smtype}{in}}
\end{record}

Solves the transient incompressible flow using algorithm based on
Characteristics Based Split (CBS, for reference see O.C.Zienkiewics
and R.L.Taylor: The Finite Element Method, 3rd volume,
Butterworth-Heinemann, 2000). At present, only semi-implicit form
of the algorithm is available and energy equation, yielding the
temperature field, is not solved.
Parameter \param{nsteps} determines number of solution
steps. Parameter \param{deltaT} is time step length used for
integration. This time step will be automatically adjusted to satisfy
integration stability limits $\Delta t \le {\frac{h}{\vert\mbf{u}\vert}}$ and $\Delta t \le {\frac{h^2}{2\nu}}$, if necessary.
Parameters \param{theta1} and \param{theta2} are integration constants, $\theta_1, \theta_2 \in \langle{\frac12}, 1\rangle$.
If \param{cmflag} is given a nonzero value, then
consistent mass matrix will be used instead of (default) lumped one.

The characteristic equations can be solved in non-dimensional form. To
enable this, the \param{scaleflag} should have a nonzero value,
and the following parameters should be provided: \param{lscale},
\param{uscale}, and \param{dscale} representing typical length,
velocity, and density scales.

Parameter \param{lstype} allows to select solver for linear system of
equations. Parameter \param{smtype} allows to select sparse matrix storage
scheme. The scheme should be compatible with solver type. See section
\ref{sparselinsolver} for further details.

\subsection{Transient incompressible flow\\SUPG/PSPG Algorithm}
\label{supgIncomp}
\begin{record}
  \recentry{\entKeywordInst{SUPG}}{\field{nsteps}{in}}
  \recentry{}{\field{deltaT}{rn}}
  \recentry{}{\field{rtolv}{rn}}
  \recentry{}{\optField{atolv}{rn}}
  \recentry{}{\optField{stopmaxiter}{in}}
  \recentry{}{\optField{alpha}{rn}}
  \recentry{}{\optField{cmflag}{in}}
  \recentry{}{\optField{deltatltf}{in}}
  \recentry{}{\optField{miflag}{in}}
  \recentry{}{\optional{\field{scaleflag}{in} \field{lscale}{in} \field{uscale}{in} \field{dscale}{in}}}
  \recentry{}{\optField{lstype}{in} \optField{smtype}{in}}
\end{record}

Solves the transient incompressible flow using stabilized formulation
based on SUPG and PSPG
stabilization terms. The stabilization provides
stability and accuracy in the solution of
advection-dominated problems and permits usage of equal-order
interpolation functions for velocity and pressure. Furthermore,
stabilized formulation significantly improves convergence rate in
iterative solution of large nonlinear systems of equations.

By changing the value $\alpha$, different methods from
``Generalized mid-point family'' can be chosen, i.e.,
Forward Euler ($\alpha=0$), Midpoint rule ($\alpha=0.5$),
 Galerkin ($\alpha=2/3$), and  Backward Euler ($\alpha=1$). Except
the first one, all the methods are implicit and require matrix inversion for solution.
Some results form an energy method analysis suggest unconditional
stability for $\alpha\ge 0.5$ for the generalized mid-point family. As
far as accuracy is concerned, the midpoint rule is to be generally preferred.

Parameter \param{nsteps} determines number of solution
steps. Parameter \param{deltaT} is time step length used for
integration. Alternatively, the load time function can be used to
determine time step length for particular solution step. The load time
function number is determined by parameter \param{deltatltf} and its
value evaluated for solution step number should yield the step length.

Parameters \param{rtolv} and \param{atolv} allow to
specify relative and absolute errors norms for residual vector.
The equilibrium iteration process will stopped when both error limits
are satisfied or when the number of iteration exceeds the value given
by parameter \param{stopmaxiter}.

If \param{cmflag} is given a nonzero value, then
consistent mass matrix will be used instead of (default) lumped one.

The algorithm allows to solve the flow of two immiscible fluids in
fixed spatial domain (currently only in 2d). This can be also used for
solving free surface problems, where one of the fluids should
represent air. To enable multi-fluid analysis, user should set parameter \param{miflag}. The supported values are described in section~\ref{materialinterfaces}.
Please note, that the initial distribution of reference fluid
volume should be provided as well as
constitutive models for both fluids.


The characteristic equations can be solved in non-dimensional form. To
enable this, the \param{scaleflag} should have a nonzero value,
and the following parameters should be provided: \param{lscale},
\param{uscale}, and \param{dscale} representing typical length,
velocity, and density scales.

Parameter \param{lstype} allows to select solver for linear system of
equations. Parameter \param{smtype} allows to select sparse matrix storage
scheme. Please note, that the present algorithm leads to a
non-symmetrical system
matrix. The scheme should be compatible with solver type. See section
\ref{sparselinsolver} for further details.

\subsection{Staggered Problem}
\label{staggeredproblem}
\begin{record}
  \recentry{\entKeywordInst{StaggeredProblem}}{(\field{nsteps}{in} \field{deltaT}{rn}) $|$ \field{timeDefinedByProb}{in}}
  \recentry{}{\field{prob1}{s} \field{prob2}{s}}
  %%\recentry{}{\optField{dtf}{in}
  \recentry{}{\optField{stepMultiplier}{rn}}
\end{record}

Represent so-called staggered analysis. This can be described as an
sequence of sub-problems, where the result of some sub-problem in the
sequence can depend on results of previous sub-problems in sequence.
Typical example is heat transfer analysis followed by mechanical
analysis taking into account the temperature field generated by the
heat transfer analysis. Similar analysis can be done when coupling
moisture transport with concrete drying strain.

The actual implementation supports only sequence of two sub-problems.
The sub-problems are described using sub-problem input files. The
syntax of sub-problem input file is the same as for standalone
problem. The only addition is that sub-problems should export their
solution fields so that they became available for subsequent
sub-problems. See the Section~\ref{StationaryTransport}.

The subproblem input files are described using
\param{prob1} and \param{prob2} parameters, which are strings
containing a path to sub-problem input files, the \param{prob1}
contains input file path of the first sub-problem, which runs first
for each solution step, the \param{prob2} contains input file path of
the second sub-problem. 

There are two options how to control a time step sequence. The first approach
uses \param{timeDefinedByProb} which uses time sequence from the corresponding subproblem. 
The subproblem may specify arbitrary loading steps and allows high flexibility.
The second approach uses the staggered problem to take control over time. Therefore any
sub-problem time-stepping parameters are ignored (even if they are
required by sub-problem input syntax) and only staggered-problem
parameters are relevant. \param{deltaT} is than a time step length used for
integration, \param{nsteps} parameter specifies
number of time steps to be solved. \param{stepMultiplier} multiplies all 
times with a given constant. Default is 1.

Note: This problem type \textbf{is included in transport module} and it
can be used only when this module is configured.
Note: All material models derived from StructuralMaterial base will
take into account the external registered temperature field, if
provided.

\subsection{Sparse linear solver parameters}
\label{sparselinsolver}
The sparselinsolverparams field has the following general syntax:\\
\begin{record}
  \recentry{\hspace{20mm}}{\optField{lstype$^M$}{in}}
  \recentry{}{\optField{smtype}{in}}
  \recentry{}{\field{solverParams$^M$}{string}}
\end{record}
where parameter \param{lstype} allows to select solver for linear system of
equations. Currently supported values are 0 (default) for direct solver
(ST\_Direct), 1 for Iterative Method Library (IML) solver (ST\_IML),
2 for Spooles direct solver, 3 for Petsc
library family of solvers, and 4 for DirectSparseSolver (ST\_DSS).
Parameter \param{smtype} allows to select sparse matrix storage
scheme. The scheme should be compatible with solver type.
Currently supported values (marked as ``id'') are summarized in table
(\ref{linsolvstoragecompattable}). The 0 value is default and selects
the symmetric skyline (SMT\_Skyline). Ther possible storage formats
include unsymmetric skyline (SMT\_SkylineU),
compressed column (SMT\_CompCol), dynamically growing compressed
column (SMT\_DynCompCol), symmetric compressed column
(SMT\_SymCompCol), spooles library storage format (SMT\_SpoolesMtrx),
PETSc library matrix representation (SMT\_PetscMtrx, a sparse
serial/parallel matrix in AIJ format), and DSS compatible matrix
representations (SMT\_DSS\_*).
The allowed \param{lstype} and \param{smtype} combinations are
summarized in the table (\ref{linsolvstoragecompattable}), together
with solver parameters related to specific solver.

\begin{table}[ht]
\begin{center}
%%\scalebox{0.50}{
\begin{tabular}{|l|c|c|c|c|c|c|}
\hline
Storage format & id & \multicolumn{5}{c|}{Sparse solver, \param{lstype}} \\
\hline
& \param{smtype} & \tiny{Direct (0)} &\tiny{IML (1)}
 &\tiny{Spooles (2)}& \tiny{Petsc (3)}& \tiny{DSS (4)}\\
\hline
\small{SMT\_Skyline}       & 0&+&+& & & \\
\small{SMT\_SkylineU}      & 1&+&+& & & \\
\small{SMT\_CompCol}       & 2& &+& & & \\
\small{SMT\_DynCompCol}    & 3& &+& & & \\
\small{SMT\_SymCompCol}    & 4& &+& & & \\
\small{SMT\_DynCompRow}    & 5& &+& & & \\
\small{SMT\_SpoolesMtrx}   & 6& & &+& & \\
\small{SMT\_PetscMtrx }    & 7& & & &+& \\
\small{SMT\_DSS\_sym\_LDL} & 8& & & & &+\\
\small{SMT\_DSS\_sym\_LL}  & 9& & & & &+\\
\small{SMT\_DSS\_unsym\_LU}&10& & & & &+\\
\hline
\end{tabular}
%%}
\caption{Solver and storage scheme compatibility.}
\label{linsolvstoragecompattable}
\end{center}
\end{table}

The solver parameters in \param{solverParams} depend on the solver
type and are summarized in table (\ref{sparsesolverparams}).

\begin{table}[ht]
\begin{center}
\begin{tabular}{|l|l|l|}
\hline
Solver type & id & Solver parameters \\
\hline
ST\_Direct  &0&\\
ST\_IML     &1& \optField{stype}{in} \field{lstol}{rn} \field{lsiter}{in}
\field{lsprecond}{in}\\
                  & &  \optField{precondattributes}{string}\\

ST\_Spooles &2&  \optField{msglvl}{in} \optField{msgfile}{s}\\
ST\_Petsc   &3& see Petsc manual, for details\footnotemark\\
ST\_DSS     &4& \\
\hline
\end{tabular}
\caption{Solver parameters.}
\label{sparsesolverparams}
\end{center}
\end{table}
\footnotetext[2]{User can set several run-time options, e.g.,
     \mbox{-ksp\_type}~$[$cg, gmres, bicg, bcgs$]$
     \mbox{-pc\_type}~$[$jacobi, bjacobi,none,ilu,...$]$
     \mbox{-ksp\_monitor} \mbox{-ksp\_rtol}~$<$rtol$>$ \mbox{-ksp\_view} \mbox{-ksp\_converged\_reason}.
     These options will override those that are default (PETSC KSPSetFromOptions() routine is called after any other customization
     routines).}
The \param{stype} allows to select particular iterative solver from IML library, currently supported values are 0 (default) for Conjugate-Gradient solver, 1 for GMRES solver. Parameter \param{lstol} represents the maximum value of residual after the
final iteration and the \param{lsiter} is maximum number of iteration for iterative solver.
The \param{precondattributes} parameters contains the optional
preconditioner parameters.
The \param{lsprecond} parameter determines the type of preconditioner to be
used. The possible values of \param{lsprecond} together with supported
storage schemes and their descriptions are summarized in table
(\ref{precondtable}).

\begin{table}[ht]
\begin{center}
\begin{tabular}{|l|l|l|l|}
\hline
Precond type & id & Compatible storage & Description and parameters \\
\hline\hline
IML\_VoidPrec &0& all & No preconditioning\\
\hline
IML\_DiagPrec &1& all & Diagonal preconditioning\\
\hline
IML\_ILUPrec  &2& SMT\_CompCol & Incomplete LU Decomposition\\
              & & SMT\_DynCompCol&with no fill up\\
\hline
IML\_ILUPrec  &3& SMT\_DynCompRow & Incomplete LU (ILUT) with\\
              & &                 &  fillup. \\
              & &                 & The \param{precondattributes} are:\\
              & &                 & \optField{droptol}{rn}
\optField{partfill}{in}.\\
              & &                 & \param{droptol} dropping
tolerance\\
              & &                 & \param{partfill} level of
fill-up\\
\hline
IML\_ICPrec   &4& SMT\_SymCompCol&Incomplete Cholesky\\
              & & SMT\_CompCol   &with no fill up\\
\hline
\end{tabular}
\caption{Preconditioning summary.}
\label{precondtable}
\end{center}
\end{table}

\subsection{Eigen value solvers}
\label{eigensolverssection}
The eigensolverparams field has the following general syntax:\\
\begin{record}
  \recentry{\hspace{20mm}}{\optField{stype$^M$}{in}}
  \recentry{}{\optField{smtype}{in}}
  \recentry{}{\field{solverParams$^M$}{string}}
\end{record}
where parameter \param{stype} allows to select solver type. Parameter \param{smtype} allows to select sparse matrix storage
scheme. The scheme should be compatible with solver type. Currently supported values of \param{stype} are summarized in table~\ref{eigenvaluesolverparamtable}.

\begin{table}[ht]
\begin{center}
%%\scalebox{0.50}{
\begin{tabular}{|l|l|l|}
\hline
Solver type & stype id & solver parameters \\
Subspace Iteration & 0 (default) & \\
Inverse Iteration& 1 & \\
SLEPc solver& 2 & requires ``smtype 7''\\
&& see also SLEPc manual \\
\hline
\end{tabular}
\caption{Eigen Solver parameters.}
\label{eigenvaluesolverparamtable}
\end{center}
\end{table}


\subsection{\Pmode{Dynamic load balancing parameters}}
\label{dynamicloadbalancing}
There are in general two basic factors causing load imbalance between individual subdomains: (i) one comming from application nature, such as switching from linear to nonlinear response in certain regions or local adaptive refinment, and (ii) external factors, caused by resourse realocation, typical for nondedicated cluster environments, where indivudual processors are shared by different applications and users, leading to time variation in allocated processing power. The load balance recovery is achieved by repartitioning of the problem domain and transferring the work (represented typically by finite elements) from one subdomain to another.
This section describes the structure and syntax of parameters related to dynamic load balancing.
The corresponding part of analysis record has the following general syntax:\\

\noindent
\begin{record}
  \recentry{\hspace{20mm}}{\optField{lbflag$^M$}{in}}
  \recentry{}{\optField{forcelb1}{in}}
  \recentry{}{\optField{wtp}{ia}}
  \recentry{}{\optField{lbstep}{in}}
  \recentry{}{\optField{relwct}{rn}}
  \recentry{}{\optField{abswct}{rn}}
  \recentry{}{\optField{minwct}{rn}}
\end{record}
where the parameters have following meaning:
\begin{itemize}
\item \param{lbflag}, when set to nonzero value activates the dynamic load balancing. Default value is zero.
\item \param{forcelb1} forces the load rebalancing after the first solution step, when set to nonzero value.
\item \param{wtp} allows to activate optional load balancing plugins. At present, the only supported value is 1, that activates nonlocal plugin, necessary for nonlocal averaging to work properly when dynamic load balancing is active.
\item \param{lbstep} rebalancing, if needed, is performed only every lbstep solution step.
Default value is 1 (recover balance after every step, if necessary).
\item \param{relwcr} sets relative wall-clock imbalance treshold. When achieved relative imbalance between wall clock solution time of individual processors is greater than provided treshold, the rebalancing procedure will be activated.
\item \param{abswct} sets absolute wall-clock imbalance treshold. When achieved absolute imbalance between wall clock solution time of individual processors is greater than provided treshold, the rebalancing procedure will be activated.
\item \param{minwct} minimum absolute imbalance to perform relative imbalance check using \param{relwcr} parameter,  otherwise only absolute check is done. Default value is 0.
\end{itemize}
At present, the load balancing support requires ParMETIS module to be configured and compiled.


%
\subsection{Error estimators and indicators}
\label{errorestimators}
The currently supported values of \param{eetype} are in table
\ref{eetypestable}.
\begin{itemize}
\item
EET\_SEI - Represents scalar error indicator.
It indicates element error based on the value of some suitable scalar
value (for example damage level, plastic strain level) obtained from the
element integration points and corresponding material model.
\item
EET\_ZZEE - The implementation of Zienkiewicz Zhu Error
Estimator. It requires the special element algorithms, which may not
be available for all element types.

Please note, that in the actual version, the error on the element level is evaluated using default integration rule. For example, in case of ZZ error estimator, the error (L2 or energy norm) is evaluated from the difference of computed and ``recovered'' stresses, which are approximated using the same interpolation functions as displacements). Therefore, in many cases, the default integration rule order is not sufficient and higher integration must be used on elements (consult element library manual and related NIP parameter).

\item
EET\_CZZSI - The implementation of combined criteria: Zienkiewicz Zhu Error Estimator for elastic regime and
scalar error indicator in non-linear regime.
\end{itemize}

\begin{table}[ht]
\begin{center}
\begin{tabular}{|l|l|}
\hline
Error estimator/indicator & \param{eetype}\\
\hline
EET\_SEI & 0 \\
EET\_ZZEE& 1 \\
EET\_CZZSI& 2\\
\hline
\end{tabular}
\caption{Supported error estimators and indicators.}
\label{eetypestable}
\end{center}
\end{table}

The sets of parameters (\param{errorestimatorparams} field) used to
configure each error estimator are different
\begin{itemize}
\item
EET\_SEI\\
\begin{record}[0.9\textwidth]
  \recentry{\hspace{2cm}}{\optField{regionskipmap}{ia}}
  \recentry{}{\field{vartype}{in}}
  \recentry{}{\field{minlim}{rn} \field{maxlim}{rn}}
  \recentry{}{\field{mindens}{rn} \field{maxdens}{rn} \field{defdens}{rn}}
  \recentry{}{\optField{remeshingdensityratio}{rn}}
\end{record}
\begin{itemize}
\item \param{regionskipmap} parameter allows to skip some regions. The
error is not evaluated in these regions and default mesh density is
used. The size of this array should be equal to number of regions and
nonzero entry indicates region to skip.
\item \param{vartype} parameter determines the type of internal
variable to be used as error indicator. Currently supported value is
1, representing damage based indicator.
\item If the indicator value is in range given by parameters (\param{minlim},
 \param{maxlim}) then the proposed mesh density is linearly
interpolated within range given by parameters (\param{mindens}, \param{maxdens}). If
indicator value is less than value of \param{minlim} parameter then
value of \param{defdens} parameter is used as
required density, if it is larger than \param{maxlim} then
\param{maxdens} is used as required density.
\item
\param{remeshingdensityratio} parameter determines the allowed ratio
between proposed density and actual density.
The remeshing is forced, whenever the actual ratio is smaller than
this value. Default value is equal to 0.80.
\end{itemize}
\item
EET\_ZZEE\\
\begin{record}[0.9\textwidth]
  \recentry{\hspace{2cm}}{\optField{regionskipmap}{ia}}
  \recentry{}{\field{normtype}{in}}
  \recentry{}{\field{requirederror}{rn}}
  \recentry{}{\field{minelemsize}{rn}}
\end{record}

\begin{itemize}
\item \param{regionskipmap} parameter allows to skip some regions. The
error is not evaluated in these regions and default mesh density is
used. The size of this array should be equal to number of regions and
nonzero entry indicates region to skip.
\item \param{normtype} Allows select the type of norm used in
evaluation of error. Default value is to use L2 norm (equal to 0),
value equal to 1 uses the energy norm.
\item \param{requirederror} parameter determines the required error to
obtain (in percents/100).
\item{minelemsize} parameter allows to set minimum limit on element size.
\end{itemize}
\item
EET\_CZZSI - combination of parameters for EET\_SEI and EET\_ZZEE; the
in elastic regions are driven using EET\_SEI, the elastic are driven
by EET\_ZZEE.
\end{itemize}


\subsection{Material interfaces}
\label{materialinterfaces}
The material interfaces are used to represent and track the position of
various interfaces on fixed grids. Typical examples include free surface,
evolving interface between two materials, etc.
Available representations include:

\begin{tabular}{|l|l|l|}
\hline
MI & miflag & Compatibility  \\
\hline
LEPlic & 0 & 2D triangular \\
LevelSet & 1 & 2D triangular \\
\hline
\end{tabular}

\begin{itemize}
\item LEPlic- representation based on Volume-Of-Fluid approach; the initial distribution of VOF fractions should be specified for each element (see element manual)\\
  \begin{record}[0.9\textwidth]
    \recentry{}{\optField{refvol}{rn}}
  \end{record}
\begin{itemize}
\item
parameter \param{refvol} allows to set initial volume of reference fluid, then the reference volume is computed in each step and printed, so the accuracy and mass conservation can be monitored.
\end{itemize}
\item
LevelSet- level set based representation\\
\begin{record}[0.9\textwidth]
  \recentry{}{\mbox{[\field{levelset}{ra} OR \field{refmatpolyx}{ra} \field{refmatpolyy}{ra}]}}
  \recentry{}{\optField{lsra}{in} \optField{rdt}{rn} \optField{rerr}{rn}}
\end{record}
\begin{itemize}
\item \param{levelset} allows to specify the initial level set values for all nodes directly. The size should be equal to total number of nodes within the domain.
\item Parameters \param{refmatpolyx} and \param{refmatpolyy} allow to initialize level set by specifying interface geometry as 2d polygon. Then polygon describes the initial zero level set, and level set values are then defined as signed distance from this polygon. Positive values are on the left side when walking along polygon. The parameter \param{refmatpolyx} specifies the x-coordinates of polygon vertices, parameter \param{refmatpolyy} y-corrdinates. Please note, that level set must be initialized, either using \param{levelset} parameter or using \param{refmatpolyx} and \param{refmatpolyy}.
\item Parameter \param{lsra} allows to select level set reinitialization algorithm. Currently supported values are 0 (no re-initialization), 1 (re-initializes the level set representation by solving $d_{\tau} = S(\phi)(1-\vert\grad d\vert)$ to steady state, default), 2 (uses fast  marching method to build signed distance level set representation).
\item Parameters \param{rdt} \param{rerr} are used to control reinitialization algorithm for \param{lsra} = 0. \param{rdt} allows to change time step of integration algorithm and parameter \param{rerr} allows to change default error limit used to detect steady state.
\end{itemize}
\end{itemize}

\subsection{Mesh generator interfaces}
\label{meshpackages}
The mesh generator interface is responsible to provide a link to
specific mesh generator. The supported values of \param{meshpackage}
parameter are
\begin{itemize}
\item
MPT\_T3D:  \param{meshpackage} = 0. T3d mesh interface. Default. Supports both 1d, 2d
(triangles) and 3d (tetrahedras) meshes. Reliable.
\item
MPT\_TARGE2:	\param{meshpackage} = 1. Interface to Targe2 2D
mesh generator.
\end{itemize}
%
\subsection{Initialization modules}
\label{InitModulesSec}
Initialization modules allow to initialize the state variables using data 
previously computed by external software. The number of initialization module records is specified in
analysis record using \param{ninitmodules} parameter (see the initial part of section \ref{_AnalysisRecord}). The general format
is the following:\\
\begin{record}
  \recentry{\entKeyword{EntType}}{\field{initfile}{string}}
\end{record}
The file name following the keyword ``initfile'' specifies
the path to the file that contains the initialization data
and should be given without quotes.

Currently, the only supported initialization module is
\begin{itemize}
\item Gauss point initialization module

\begin{record}[0.9\textwidth]
  \recentry{GPInitModule}{\field{initfile}{string}}  \end{record}
\begin{itemize}
\item
Each Gauss point is represented by one line in the initialization file.
\item
The Gauss points should be given in a specific order, based on the element number and the Gauss point number, in agreement with the mesh specified in later sections.
\item
Each line referring to a Gauss point should contain the following data:

\begin{record}
  \recentry{}{\field{elnum}{in} \field{gpnum}{in} \field{coords}{ra} \field{ng}{in}}
  \recentry{}{\field{var\_1\_id}{in} \field{values\_1}{ra}}
  \recentry{}{ ...}
  \recentry{}{\field{var\_ng\_id}{in} \field{values\_ng}{ra}}
\end{record}
   \item \param{elnum} is the element number
   \item \param{gpnum} is the Gauss point number
   \item \param{coords} are the coordinates of the Gauss point
   \item \param{ng} is the number of groups of variables that will follow
   \item \param{var\_1\_id} is the identification number of variable group number 1 (according to the definitions in internalstatetype.h)
   \item \param{values\_1} are the values of variables in group number 1
   \item \param{var\_ng\_id} is the identification number of variable group number ng
   \item \param{values\_ng} are the values of variables in group number ng
\item Example:\\ \mbox{``37 4 3 0.02 0.04 0.05 3 52 1 0.23 62 1 0.049 1 6 0 -2.08e+07 0 0 0 0''} means that Gauss point number 4 of element number 37 has coordinates $x=0.02$, $y=0.04$ and $z=0.05$ and the initial values are specified for 3 groups of variables;\\ 
the first group (variable ID 52) is of type IST\_DamageScalar (see internalstatetype.h) and contains 1 variable (since it is a scalar) with value 0.23;\\
 the second group (ID 62) is of type IST\_CumPlasticStrain
and contains 1 variable with value 0.049;\\
 the third group is of type IST\_StressTensor
and contains 6 variables (stress components $\sigma_x$, $\sigma_y$, etc.) with values
0, -2.08e+07, 0, 0, 0, 0
\end{itemize}
\end{itemize}

\subsection{Export modules}
\label{ExportModulesSec}
Export modules allow to export computed data into external software for
post-processing. The number of export module records is specified in
analysis record using \param{nmodules} parameter (see the initial part of section \ref{_AnalysisRecord}). The general format
is the following:\\
\begin{record}
  \recentry{\entKeyword{EntType}}{[\fieldnotype{tstep\_all}]}
  \recentry{}{\optField{tstep\_step}{in}}
  \recentry{}{\optField{tsteps\_out}{rl}}
  \recentry{}{\optField{subtsteps\_out}{in}}
  \recentry{}{[\fieldnotype{domain\_all}]}
  \recentry{}{\optField{domain\_mask}{in}}
\end{record}

To select all solution steps, in which output will be performed, use
\param{tstep\_all}. To select each \param{tstep\_step}-nth step, use
\param{tstep\_step} parameter. In order to select only specific
solution steps, the \param{tsteps\_out} list can be specified,
supplying solution step number list in which output will be done.
To select output for all domain of the problem the \param{domain\_all}
keyword can be used. To select only specific domains,
\param{domain\_mask} array can be used, where the values of the array
specify the domain numbers to be exported.
If the parameter \param{subtsteps\_out} = 1, it turns on the export of intermediate results, 
for example during the substepping or individual equilibrium iterations. This option requires support from the solver.

Currently, the supported export modules are following
\begin{itemize}
\item VTK export, \textbf{DEPRECATED - Use VTKXML}

\begin{record}[0.9\textwidth]
  \recentry{\entKeywordInst{vtk}}{\optField{vars}{ia}}
  \recentry{}{\optField{primvars}{ia}}
  \recentry{}{\optField{cellvars}{ia}}
  \recentry{}{\optField{stype}{in}}
  \recentry{}{\optField{regionstoskip}{ia}}
\end{record}


\begin{record}[0.9\textwidth]
  \recentry{\entKeywordInst{vtkxml}}{\optField{vars}{ia}}
  \recentry{}{\optField{primvars}{ia}}
  \recentry{}{\optField{cellvars}{ia}}
  \recentry{}{\optField{ipvars}{ia}}
  \recentry{}{\optField{stype}{in}}
  \recentry{}{\optField{regionsets}{ia}}
  \recentry{}{\optField{timeScale}{rn}}
\end{record}

\begin{itemize}
\item The vtk module is obsolete, use vtkxml instead. Vtkxml allows to export results recovered on region by region basis and has more features.

\item The array \param{vars} contains identifiers for those internal
variables which are to be exported. These variables will be smoothed and transfered to nodes. The id values are defined by
InternalStateType enumeration, which is defined in include file
``src/oofemlib/internal\-statetype.h''.

\item The array \param{primvars}
contains identifiers of primary variables to be exported.
The possible values correspond to the values of enumerated type
UnknownType, which is again defined in ``src/oofemlib/unknown\-type.h''.
Please note, that the values corresponding to enumerated type values
start from zero, if not specified directly and that not all values are
supported by particular material model or analysis type.

\item The array \param{cellvars} contains identifiers of constant variables defined
on an element (cell), e.g. a material number. Identifier numbers are specified in
``src/oofemlib/internalstatetype.h''.

\item The array \param{ipvars} contains identifiers for those internal
variables which are to be exported. These variables will be directly exported (no smoothing) as point dataset, where each point corresponds to individual integration point. A separate vtu file for these raw, point data will be created. 
The id values are defined by
InternalStateType enumeration, which is defined in include file
``src/oofemlib/internal\-statetype.h''.


\item The parameter \param{stype} allows to select smoothing procedure for
internal variables, which is used to compute nodal values from values
in integration points. The supported values are $0$ for simple nodal averageing (generally supported only by triangular and tetrahedral elements), $1$ for Zienkiewicz Zhu recovery (default), and $2$ for Superconvergent Patch Recovery (SPR, based on least square fitting).

\item The export is done on region basis, on each region, the nodal recovery is performed independently and results are exported in a separate piece. This allows to take into account for discntinuities, or to export variables defined only by particular material model. The region volumes are defined using sets containing individual elements. By default the one region is created, containing all element in the problem domain. The optional parameter \param{regionsets} allows to use user-defined.  The individual values refer to numbers (ids) of domain sets. Note, that regions are determined solely using elements.

{\footnotesize vtkxml tstep\_all cellvars 1 46 vars 1 1 primvars 1 1 stype 2 regionsets 2 1 2}

\item \param{timeScale} scales time in output. In transport problem, basic units are seconds. Setting timeScale = 2.777777e-4 (=1/3600.) converts all time data in vtkXML from seconds to hours.

\end{itemize}

By default vtk and vtkxml modules perform recovery over the whole domain. The VTKXML module can operate in region-by-region mode (see \param{nvr} and \param{vrmap} parameters). In this case, the smoothing is performed only over particular virtual region, where only elements in this virtual region participate. 

\item Homogenization of stresses and strains in the global coordinate system. Corresponding stress and strain components are summed and averaged over the volume. It is possible to select material numbers from which the averaging occurs. The averaging works for 3D domains with an extension to trusses. A truss is considered as a volume element with oriented stress and strain components along the truss axis. The transformation to global components occurs before averaging.

\begin{record}[0.9\textwidth]
  \recentry{hom}{\optField{scale}{rn}}
  \recentry{}{\optField{MatNum}{ia}}
\end{record}
\begin{itemize}

\item The parameter \param{scale} multiplies all averaged stresses and strains. \param{scale}=1 by default.

\item An integer array \param{MatNum} specifies which material numbers are taken into account. All material numbers are averaged by default.
\end{itemize}

\item Gauss point export is useful if one needs to plot a certain variable
(such as damage) as a function of a spatial coordinate using tools like
gnuplot. It generates files with data organized in columns, each row
representing one Gauss point. In this way, one can plot e.g.\ the damage
distribution along a one-dimensional bar.

\begin{record}[0.9\textwidth]
  \recentry{gpexportmodule}{\optField{vars}{ia}}
  \recentry{}{\optField{ncoords}{in}}
\end{record}

\begin{itemize}
\item The array \param{vars} contains identifiers for those internal
variables which are to be exported. The id values are defined by
InternalStateType enumeration, which is defined in include file
``src/oofemlib/internal\-statetype.h''.
\item Parameter \param{ncoords} specifies the number of spatial coordinates to be exported at each Gauss point. Depending on the spatial dimension of the
domain, the points can have one, two or three coordinates. If  \param{ncoords}
is set to -1, only those coordinates that are actually used are exported.
 If  \param{ncoords} is set to 0, no coordinates are exported.
If  \param{ncoords}
is set to a positive integer, exactly \param{ncoords} coordinates are exported.
If  \param{ncoords} exceeds the actual number of coordinates, the actual
coordinates are supplemented by zeros. For instance, if we deal with a 2D
problem, the actual number of coordinates is 2. For  \param{ncoords}=3,
the two actual coordinates followed by 0 will be exported.
For  \param{ncoords}=1, only the first coordinate will be exported.
\end{itemize}
The Gauss point export module creates a file with extension ``gp''
after each step for which the output is performed. This file contains
a header with lines starting by the symbol \#, followed by the actual data
section.
Each data line corresponds to one Gauss point and contains the following
data:
\begin{enumerate}
\item element number,
 \item    material number,
 \item     Gauss point number,
  \item   contributing volume around Gauss point,
\item Gauss point global coordinates (written as a real array of length \param{ncoords}),
\item internal variables according to the specification in \param{vars} (each written as a real array of the corresponding length).
\end{enumerate}
Example:\\
``GPExportModule 1 tstep\_step 100 domain\_all ncoords 2 vars 5 4 13 31 64 65''
\\
means that the *.gp file will be written after each 100 steps and will contain
for each of the Gauss points in the entire domain its 2 coordinates and also internal
variables of type 4, 13, 31, 64 and 65, which are the strain tensor, damage tensor, maximum equivalent strain level, stress work density and dissipated work density. Of course, the material model must be able to deliver such variables. 
The size of the strain tensor depends on the spatial dimension, and the size
of the damage tensor depends on the spatial dimension and type of model
(e.g., for a simple isotropic damage model it will have just 1 component
while for an anisotropic damage model it may have more). The other variables
in this example are scalars, but they will be written as arrays of length 1,
so the actual value will always be preceded by ``1'' as the length of the array. Since certain internal variables have the meaning of densities (per unit volume
or area, again depending on the spatial dimension), it is useful to have
access to the contributing volume  of the Gauss point. The product of this
contributing volume and the density gives an additive contribution to the
total value of the corresponding variable. This can be exploited e.g.\ to
evaluate the total dissipated energy over the entire domain.
\end{itemize}

\section{Domain record(s)}
\label{_DomainRecord}
This set of records describes the whole domain and its type. Depending
on the type of problem, there may be one or several domain records. If not
indicated, one domain record is default for all problem types.

The domain type is used to resolve the
default number of DOFs in node and their physical meaning. Format is following\\
\begin{record}
  \recentry{\fieldnotype{domain}}{\entKeyword{domainType}}
\end{record}\\
The \entKeyword{domainType} can be one from the following
\begin{itemize}
\item The \entKeywordInst{2dPlaneStress} and \entKeywordInst{2d-Truss}
modes declare two default dofs per node (u-displacement, v-displacement),
\item The \entKeywordInst{3d} mode declares three default dofs per
node (u-displacement, v-displacement, w-displacement),
\item The \entKeywordInst{2dMindlinPlate} mode declares three default
dofs per node (w-displacent, u-rotation, v-rotation). Strain vector contains
$\kappa_{xx}$, $\kappa_{yy}$, $\kappa_{xy}$, $\gamma_{xz}$, $\gamma_{yz}$. Stress vector
contains $m_{xx}$, $m_{yy}$, $m_{xy}$, $q_{xz}$, $q_{yz}$.

\item The \entKeywordInst{3dShell} mode declares six default dofs
per node (displacement and rotation along each axis).
\item The \entKeywordInst{2dBeam} mode declares three default dofs per
node (u-displacement, w-displacement, v-rotation).
\item The \entKeywordInst{2dIncompFlow} mode declares three default
  dofs per node (u-velocity, v-velocity, and pressure).
The default number of dofs per node as well as their physical meaning
can be overloaded in particular dof manager record (see section
\ref{_NodeElementSideRecords}).

The further records describe particular domain components -
OutputManagers, DofManagers, Elements, CrossSection models, Material
Models, Boundary and Initial Conditions and Load time functions.

\end{itemize}


\subsection{Output manager record}
\label{_OutputManagerRecord}
The output manager controls output. It can filter output to specific
solution steps, and within these selected steps allows also to filter
output only to specific dof managers and elements. The format of
output manager record is \\
\begin{record}
  \recentry{\entKeywordInst{OutputManager}}{[\fieldnotype{tstep\_all}]}
  \recentry{}{\optField{tstep\_step}{in}}
  \recentry{}{\optField{tsteps\_out}{rl}}
  \recentry{}{[\fieldnotype{dofman\_all}]}
  \recentry{}{\optField{dofman\_output}{rl}}
  \recentry{}{\optField{dofman\_except}{rl}}
  \recentry{}{[\fieldnotype{element\_all}]}
  \recentry{}{\optField{element\_output}{rl}}
  \recentry{}{\optField{element\_except}{rl}}
\end{record}
To select all solution steps, in which output will be performed, use
\param{tstep\_all}. To select each \param{tstep\_step}-nth step, use
\param{tstep\_step} parameter. In order to select only specific
solution steps, the \param{tsteps\_out}list can be specified,
supplying solution step number list in which output will be done.
The combination of \param{tstep\_step} and
\param{tsteps\_out} parameters is allowed.

Output manager allows also to filter output to only specific dof
managers and elements. If these specific members are selected, the
output happens only in selected solution steps.
The \param{dofman\_all} and \param{element\_all} parameters select
all dof managers or elements respectively. Parameter arrays
\param{dofman\_output} and \param{element\_output} allow to select
only specific members. Numbers of selected members are then contained
in \param{dofman\_output} or \param{element\_output} lists
respectively. The previously selected members can be explicitly
de-selected by specifying their component numbers in \param{dofman\_except} or
\param{element\_except} lists. A few examples:\\
\param{dofman\_output} \{1 3\}~~~prints nodes 1,3\\
\param{dofman\_output} \{(1 3)\}~~~prints nodes 1,2,3\\
\param{element\_output} \{1 3\}~~~prints elements 1,3\\
\param{element\_output} \{(1 3)\}~~~prints elements 1,2,3\\
\param{element\_output} \{(1 3) 5 6\}~~~prints elements 1,2,3,5,6\\

\subsection{Components size record}
\label{_ComponentsSizeRecord}
This record describes the number of components in related domain. The
particular records will follow immediately in input file. The general format is:\\
\begin{record}
  \recentry{}{\field{ndofman}{in}}
  \recentry{}{\field{nelem}{in}}
  \recentry{}{\field{ncrosssect}{in}}
  \recentry{}{\field{nmat}{in}}
  \recentry{}{\field{nbc}{in}}
  \recentry{}{\field{nic}{in}}
  \recentry{}{\field{nltf}{in}}
  \recentry{}{\optField{nbarrier}{in}}
\end{record}
where \param{ndofman} represents number of dof managers (e.g. nodes) and their associated records,
\param{nelem} represents number of elements and their associated records, \param{ncrosssect} is
number of cross sections and their records, \param{nmatdnMat}{} is number of material
models and their records, \param{nbc}{} represents number of boundary
conditions (including loads) and their
records, \param{nic} parameter determines the number of initial
conditions, and \param{nltf} represents number of time functions and
their associated records. The optional parameter \param{nbarrier}
represents the number of nonlocal barriers and their records. If not
specified, no barriers are assumed.

\subsection{Dof manager records}
\label{_NodeElementSideRecords}
These records describe individual DofManager records (i.e. nodes or element sides (if they manage some DOFs)). The general format is following:

\noindent
\begin{record}
  \recentry{\entKeyword{DofManagerType}}{\componentNum}
  \recentry{}{[\field{load}{ra}]}
  \recentry{}{[\field{ndofs}{in} \field{DofIDMask}{ia}]}
  \recentry{}{\optField{bc}{ia}}
  \recentry{}{\optField{ic}{ia}}
  \recentry{}{[\field{doftype}{ia} \field{masterMask}{ia}]}
  \recentry{}{\PoptFieldnotype{shared} $|$ \PoptFieldnotype{remote} $|$ \PoptFieldnotype{null}}
  \recentry{}{\PoptField{partitions}{ia}}
\end{record}
The order of particular records is optional, the dof manager number is determined by \componentNum\ parameter.
The numbering of individual dof managers is arbitrary, it could be even non-continuous. In this context, one could think of dof manager number as a label that is assigned to individual dof manager and by which the dof manager is referenced.
\Pmode{In parallel mode, the label represents a global id across all partitions.}


The applied primary (Dirichlet) boundary conditions are specified using
"bc" record, while natural boundary conditions using "load" parameter.
\begin{itemize}
\item The size of "bc" array (primary bc) should be equal to number of DOFs in dof manager and i-th value relates to i-th DOF - the ordering and physical meaning of DOFs is determined by domain record and can be optionally specified for each dof manager individually (see next paragraph). The values of this array are corresponding boundary condition record numbers or zero, if no primary bc is applied to corresponding DOF. The compatible boundary condition type are required: primary conditions require "BoundaryCondition" records.
\item The load "array" contains record numbers of natural boundary conditions that are applied. The required record type for natural condition is "NodalLoad". The actual value is the summation of all contributions, if more than one natural bc is applied. See section on boundary conditions for the syntax. Please note, that the values of natural bc for individual DOFs are specified in its record, not in dofmanager record.
\end{itemize}

By default, if "bc" and/or "load" parameters are omitted, no primary
and/or natural bc are applied. Analogously, initial conditions are represented using \param{ic} array. The size of \param{ic} array should be equal to number of DOFs in dof manager. The values of this array are corresponding initial condition record numbers or zero, if no initial condition is applied to corresponding DOF (in this case zero value is assumed as value of initial condition).

By default, the number of DOFs per node or side  and their physical meanings are determined by domain record (\entKeyword{domainType} keyword).
If it is necessary to have
different number of DOFs, then the \param{ndofs} attribute
determines number of DOFs in DofManager, and their physical meaning is
determined by \param{DofIDMask} array. Size of this array
should be equal to \param{ndofs} value. Each item of \param{DofIDMask}
array describes the physical meaning of corresponding DOF in dof manager. Currently
the following values are supported: \{u-displacement=1, v-displacement=2,
 w-displacement=3, u-rotation=4, v-ro\-ta\-tion=5, w-rotation=6, u-velocity=7, v-velocity=8, w-velocity=9,
temperature=10, pressure=11, special dofs for gradient-type constitutive models=12 and 13, mass concentration=14, special dofs for extended finite elements (XFEM)=15--30\}.
\textbf{It is not allowed to have two DOFs with the same physical meaning in the same DofManager.}

Parameters \param{dofType} and \param{masterMask} allows to connect some dof manager's dofs
(so-called ``slave'' dofs)
to corresponding dof (according to their physical meaning) of another
dof manager (so-called ``master'' dof). The master slave principle
allows for example simple modeling of structure hinges, where multiple
elements are connected by introducing multiple nodes (with same
coordinates) sharing the same displacement dofs and each one possessing
their own rotational dofs.
Parameter
\param{dofType} determines the type of (slave) dof to
create. Currently supported values are 0 for master DOF, 1 for simpleSlave DOF (linked to another single master DOF), and 2 for general
slave dof, that can depend on different DOFs belonging to different
dof managers.
If \param{dofType} is not specified, then by default all DOFs are
created as master DOFs. If provided, masterMask is also required.
The meaning of \param{masterMask} parameter is
depending on type of particular dofManager, and will be described in
corresponding sections.

\Pmode{
The \param{shared} indicates, that dofmanager is shared by
neighboring partitions. The contributions from all contributing
domains are summed. Typical for node cut algorithm (see figures \ref{nodecut-lm}
and \ref{nodecut-nlm}).

Remote DofManager is indicated by \param{remote} parameter.
Then DofManager in active domain is only mirror of some remote
DofManager and it is necessary to copy remote values into local
ones. Typical for element cut (see fig. \ref{elmentcut-lm}).
The \param{null} parameter indicates so-called null DofManager. The
null DofManager should be shared only by remote elements (these are
only introduced for nonlocal constitutive model to allow effective
local averaging, so only local material value to be averaged are
transferred for these remote elements). Null nodes are therefore used
only for computing real integration point coordinates of remote
elements and there is no reason to maintain their unknowns (they have
no equation number assigned, see fig. \ref{nodecut-nlm}). They do not
contribute to local partition governing equation.
Only one of the \param{null} \param{remote} \param{shared} parameters can
be used for particular DofManagers. If no one is used, the DofManager
is maintained as local for particular partition.

The list of remote partitions sharing corresponding DofManager or list
containing  remote partition containing remote DofManager counterpart is
specified using \param{partitions} parameter. The local partition
should not be included in the list.
The slaves are allowed, but masters have to be in the same
partition. The masters can be again remote copies.
}


Supported DofManagerType keywords are
\begin{itemize}
\item Node record

\begin{record}[0.9\textwidth]
  \recentry{\entKeywordInst{Node}}{\field{coords}{ra}}
  \recentry{}{[\field{lcs}{ra}]}
\end{record}
Represent an abstraction for finite element node.
The node coordinates in space (given by global coordinate system) are described
using \param{coords} attribute. This array contains x, y and possibly z
(depends on problem under consideration) coordinate of node.
By default, the coordinate system in node is global coordinate system.
User defined local coordinate system in node is described using \param{lcs} array. This
array contains six numbers, where the first three numbers represent
a directional vector of the local x-axis, and the next three numbers represent
a directional vector of the local y-axis. The local z-axis is determined
using a vector product. A right-hand coordinate system is assumed.
If user defined local coordinate system in node is specified, then the
boundary conditions and applied loading are specified in this local coordinate system.
The reactions and displacements are also in \param{lcs} system at the output.

The node can create only master DOFs and SimpleSlave DOFs, so the
allowable values of \param{dofType} array are in range {0,1}.
For the Node dof manager, the \param{masterMask} is the array of size
equal to number of DOFs, and the i-th value determines the
master dof manager, to which i-th dof is directly linked (the dof with
same physical meaning are linked together).
The local coordinate
system in node with same linked dofs is supported, but it should be exactly the
same as on master.


\item Rigid arm record

\begin{record}[0.9\textwidth]
  \recentry{\entKeywordInst{RigidArmNode}}{\field{coords}{ra}}
  \recentry{}{\field{master}{in}}
  \recentry{}{\optField{masterMask}{ia}}
\end{record}
Represent node connected to other node (called master) using rigid
arm. Rigid arm node posses no degrees of freedom - all dofs are mapped to master
dofs. The introduction of rigid arm connected nodes allows to avoid very
stiff elements used for modelling the rigid-arm connection.
The rigid arm node maps its dofs to master dofs using simple transformations
(small rotations are assumed). Therefore, the contribution to rigid
arm node are localized directly to master related equations.
{\em The rigid arm node can not have its own boundary or initial conditions, they are
determined completely from master dof conditions. Currently it is
possible to map only certain dofs - see \param{dofType}. Linked DOFs
should have dofType value equal to 2, non-linked (primary) DOFs 0.}

Rigid arm node can be loaded
independently of master. The node coordinates in space (given by
global coordinate system) are described using \param{coords}
field. This array contains x, y and possibly z (depends on problem
under consideration) coordinate of node. The \param{master} parameter is the
master node number, to which rigid arm node dofs are mapped.
{\em The current implementation allows chaining of rigid arm nodes.}
The optional parameter \param{masterMask} allows to specify how
particular mapped DOF depends on master DOFs. The size of \param{masterMask} array
should be equal to number of DOFs. For all linked DOFs (with
corresponding dofType value equal to 2) the corresponding value of
\param{masterMask} array should be 1.

\item Hanging node

\begin{record}[0.9\textwidth]
  \recentry{\entKeywordInst{HangingNode}}{\field{coords}{ra}}
  \recentry{}{\field{dofType}{in}}
  \recentry{}{\optField{masterElement}{in}}
  \recentry{}{\optField{masterRegion}{in}}
\end{record}

Hanging node is connected to an a master element using generalized interpolation.
Hanging node posses no degrees of freedom (except unlined dofs) - all values are
interpolated from corresponding master elements and its DOFs.
arbitrary FE mesh of concrete specimen or to facilitate the local
refinement of FE mesh. The hanging nodes can be in a chain.

The contributions of hanging node are localized directly to master related equations.
The hanging node can have its own boundary or initial conditions, but
only for primary unlinked DOFs. For linked DOFs, these conditions are
determined completely from master DOF conditions.
The local coordinate system should be same for all master nodes.
The hanging node can be loaded independently of its master.

Values of array \param{dofType} can have following values:
0-primary DOF, 2-linked DOF.

The value of \param{masterElement} specifies the element number to which the hanging node is attached.
The node can be attached to any arbitrary coordinate within the master element.
The element must support the necessary interpolation classes.
The same interpolation for unknowns and geometry is assumed.

The no (or -1) value for \param{masterElement} is supplied, then the node will locate the element closest to its coordinate.
If no (or zero) value for \param{masterRegion} is supplied, then all regions will be searched,
otherwise only the elements in cross section with number \param{masterRegion}.
If \param{masterElement} is directly supplied \param{masterRegion} is unused.

\item Slave node

\begin{record}[0.9\textwidth]
  \recentry{\entKeywordInst{SlaveNode}}{\field{coords}{ra}}
  \recentry{}{\field{dofType}{in}}
  \recentry{}{\field{masterDofMan}{ia}}
  \recentry{}{\field{weights}{ra}}
\end{record}

Works identical to hanging node, but the weights (\param{weights}) are not computed from any element,
but given explicitly, as well as the connected dof managers (\param{masterDMan}).

\item Element side

\begin{record}[0.9\textwidth]
  \recentry{\entKeywordInst{ElementSide}}{}
\end{record}
Represents an abstraction for element side, which holds some unknowns.
\end{itemize}

\subsection{Element records}
\label{_ElementsRecords}
These records specify a description of particular elements. The
general format is following:

\noindent
\begin{record}
  \recentry{\entKeyword{ElementType}}{\componentNum}
  \recentry{}{\field{mat}{in} \field{crossSect}{in} \field{nodes}{ia}}
  \recentry{}{\optField{bodyLoads}{ia} \optField{boundaryLoads}{ia}}
  \recentry{}{\optField{activityltf}{in} \optField{lcs}{ra}}
  \recentry{}{\PoptField{partitions}{ia} \PoptFieldnotype{remote}}
\end{record}

The order of element records is optional, the element number is determined by \componentNum\ parameter.
The numbering of individual elements is arbitrary, it could be even non-continuous. In this context, one could think of element number as a label that is assigned to individual elements and by which the element is referenced.
\Pmode{In parallel mode, the label represents a global id across all partitions.}

Element
material is described by parameter \param{mat}, which contains corresponding
material record number. Element cross
section is determined by cross section  with  \param{crossSect}
record number. Element dof managers (nodes, sides, etc.) defining element geometry are specified using
\param{nodes} array.

Body load acting on element is specified using \param{bodyLoads} array. Components
of this array are corresponding load record numbers. The loads should
have the proper type (body load type), otherwise error will be generated.

Boundary load acting on element boundary is specified using
\param{boundaryLoads} array. The format of this array is
\begin{displaymath}
2\cdot size \quad lnum(1)~id(1)~\dots~lnum(size)~id(size),
\end{displaymath}
where $size$ is total number of loadings applied to element,
$lnum(i)$ is the applied load number, and $id(i)$
is the corresponding entity number, to which the load is applied (for example
a side or a surface number). The entity numbering is element dependent and
is described in element specific sections. The applied loads must be
of proper type (boundary load type), otherwise error is generated.

The support for element insertion and removal during the analysis is provided. One can specify optional time function (identified by its id using \param{activityltf} parameter). The nonzero value of this time function indicates, whether the element is active (nonzero value, the default) or inactive (zero value) at particulat solution step. Tested for structural and transport elements. This feature allows considering temperature evolution of layered casting of concrete, where certain layers needs to be inactive before they are cast. See a corresponding example in oofem
tests how to enforce hydrating material model, boundary conditions and element activity acting concurrently.

Orientation of local coordinates can be specified using \param{lcs} array. This array contains six numbers,
where the first three numbers represent a directional vector of local
x-axis, and the next three numbers represent a directional vector of local
y-axis. The local z-axis is determined using the vector product. The \param{lcs} array on the element is particularly useful for modeling of orthotropic materials which follow the element orientation. On a beam or truss element, the \param{lcs} array has no effect and the 1D element orientation is aligned with the global $xx$ component.

\Pmode{
The \param{remote} forces the element to be remote element. Remote
element does not contribute to local partition governing equation.
They are introduced in order to implement band of elements involved
in computation of nonlocal variables (see fig. \ref{nodecut-nlm} illustrating
this approach for node-cut partitioning). They role is to provide
local mirror of corresponding remote partition element integration point values which undergo
nonlocal averaging on local partition.
%This mirroring will lead to
%minimized communication pattern when exchanging remote element data,
%because values are transferred only once, instead of fine grain
%communication when each local element possibly requests its remote
%data over network.
If not used, element is assumed to be local partition element.
When \param{remote} is used, the \param{partitions} parameter should
contain remote partition number, where
corresponding element is local (this array should have size equal to one).
}


Available material models, their outline and
corresponding parameters are described in separate \textbf{Element Library Manual.}

\subsection{Cross section records}
\label{_CrossSectionRecords}
These records specify a cross section model descriptions. The general format is
following:

\begin{record}[0.9\textwidth]
  \recentry{\entKeyword{CrossSectType}}{\componentNum}
\end{record}

The order of particular cross section records is optional, cross section model number is determined by \componentNum\ parameter.
The numbering should start from one and should end at n, where n is the number of records.

The crossSectType keyword can be one from following possibilities
\begin{itemize}
\item 
Integral cross section with constant properties\\
\begin{record}[0.9\textwidth]
  \recentry{\entKeywordInst{SimpleCS}}{\optField{thick}{rn} \optField{width}{rn} \optField{area}{rn}}
  \recentry{}{\optField{iy}{rn} \optField{iz}{rn} \optField{ik}{rn}}
  \recentry{}{\optField{shearareay}{rn} \optField{shearareaz}{rn} \field{beamshearcoeff}{rn}}
\end{record}
Represents integral type of cross section model. In current
implementation, such cross section is described using cross section
thick (\param{thickVal}) and width (\param{widthVal}). For some
problems (for example 3d), the corresponding volume and cross section dimensions are
determined using element geometry, and then you can omit some (or all) parameters (refer to documentation of individual elements for required cross section properties). Parameter \param{area} allows to set cross section area, parameters \param{iz}, \param{iz}, and \param{ik} represent 
inertia moment along y and z axis and torsion inertia moment. Parameter \param{beamshearcoeff} allows to set shear correction factor, or equivalent 
shear areas (\param{shearareay} and \param{shearareaz} parameters) can be provided. 
These cross section properties are assumed to be defined in local coordinate system of element.\\

\item
Integral cross section with variable properties\\
\begin{record}[0.9\textwidth]
  \recentry{\entKeywordInst{VariableCS}}{\optField{thick}{expr} \optField{width}{expr} \optField{area}{expr}}
  \recentry{}{\optField{iy}{expr} \optField{iz}{expr} \optField{ik}{expr}}
  \recentry{}{\optField{shearareay}{expr} \optField{shearareaz}{expr} }
\end{record}
Represents integral type of cross section model, where individual cross section parameters can be expressed as an arbitrary function of global coordinates x,y,z. Similar to SimpleCS, for some problems (for example 3d), the corresponding volume and cross section dimensions are
determined using element geometry, then you can omit many (or some) parameters (refer to documentation of individual elements for required cross section properties). 
Parameter \param{area} allows to set cross section area, parameters \param{iz}, \param{iz}, and \param{ik} represent 
inertia moment along y and z axis and torsion inertia moment. Parameters (\param{shearareay} and \param{shearareaz} determine shear area, which is required by beam and plate elements. All cross section properties are assumed to be defined in local coordinate system of element.\\

\item
Layered cross section \\
\begin{record}[0.9\textwidth]
  \recentry{\entKeywordInst{LayeredCS}}{\field{nLayers}{in}}
  \recentry{}{\field{LayerMaterials}{ia}}
  \recentry{}{\field{Thicks}{ra} \field{Widths}{ra}}
  \recentry{}{\field{mid\-Surf}{rn}}
\end{record}
Represents the layered cross section model, based on
geometrical hypothesis, that cross sections remain planar after
deformation. Number of layers is
determined by \param{nLayers} parameter. Materials for each
layer are specified by \param{LayerMaterials} array. For each layer is
necessary to input
geometrical characteristic, thick - using \param{Thicks} array, and
width - using
\param{Widths} array. Position of mid surface is determined by its
distance from
bottom of cross section using \param{mid\-Surf} parameter (normal and
momentum forces are then computed with regard to it's position).
Elements using this cross section model must implement layered cross
section extension. For information see element library manual.
\item
Fibered cross section\\
\begin{record}[0.9\textwidth]
  \recentry{\entKeywordInst{FiberedCS}}{\field{nfibers}{in} \field{fibermaterials}{ia}}
  \recentry{}{\field{thicks}{ra} \field{widths}{ra} \field{thick}{rn} \field{width}{rn}}
  \recentry{}{\field{fiberycentrecoords}{ra} \field{fiberzcentrecoords}{ra}}
\end{record}
Cross section represented as a set of rectangular fibers. It is based on
geometrical hypothesis, that cross sections remain planar after
deformation (3d generalization of layered approach for beams).
Paramater \param{nfibers} determines the number of fibers that together form the overall cross section.
The model requires to specify a material model corresponding to particular fiber using \param{fibermaterials} array. This array should contain for each fibre corresponding material model number (the material model specified on element level has no meaning in this particular case).
\textbf{The geometry of cross section is determined from fiber dimensions and fiber positions, all input in local coordinate system of the beam (yz plane).} The thick and width of each fiber are determined using \param{thicks} and \param{widths} arrays. The overall thick and width are specified using parameters \param{thick} and \param{width}. Positions of particular fibers are specified by providing coordinates of center of each fiber using \param{fiberycentrecoords} array for y-coordinates and \param{fiberzcentrecoords} array for z-coordinates.
%%\item \entKeywordInst{HeatCS} \field{thick}{rn}
%%\field{width}{rn}\\
%%Represents cross section abstraction for heat transfer problems.
%%The input parameters are: thick - determined using \param{thick} parameter
%%and width - determined using \param{width} parameter of cross section.
\end{itemize}

\subsection{Material type  records}
\label{_MaterialTypeRecords}
These records specify a material model  description. The general format is
following:

\noindent
\begin{record}
  \recentry{\entKeyword{MaterialType}}{\componentNum\ \field{d}{rn}}
\end{record}

The order of particular material records is optional, the material number is determined by \componentNum\ parameter.
The numbering should start from one and should end at n, where n is the number of records.
Material density is compulsory parameter and it's value is given by
\param{d} parameter.

Available material models, their outline and
corresponding parameters are described in separate \textbf{Material Library Manual}.

\subsection{Nonlocal barrier records}
\label{_NonlocalBarrierRecords}
Nonlocal material models of integral type are based on replacement of
certain suitable local quantity in local constitutive law by their
nonlocal counterparts, that are obtained as weighted average over
some characteristic volume. The weighted average is computed as a sum
of a remote value multiplied by weight function value. The weight
function typically depend on a distance between remote and receiver
points and decreases with increasing distance. In some cases, it is
necessary to disregard mutual interaction between some points (for
example if they are on the opposite sides of a thin notch, which
prevents the nonlocal interactions to take place). The barriers are
the way how to introduce these constrains. The barrier represent a
curve (in 2D) or surface (in 3D). When the line connecting receiver and
remote point intersects a barrier, the barriers is activated and the
corresponding interaction is not taken into account.

Currently, the supported barrier types are following:
\begin{itemize}
\item
  Polyline barrier

\begin{record}[0.9\textwidth]
  \recentry{\entKeywordInst{polylinebarrier}}{\componentNum\ \field{vertexnodes}{ia}}
  \recentry{}{\optField{xcoordindx}{in} \optField{ycoordindx}{in}}
\end{record}
This represents a polyline barrier for 2D problems. Barrier is a
polyline, defined as a sequence of nodes representing vertices.
The vertices are specified using parameter \param{vertexnodes} array,
which contains the node numbers. The optional parameters
\param{xcoordindx} and \param{ycoordindx} allow to select the plane
(xy, yz, or xz), where the barrier is defined. The \param{xcoordindx} is the
first coordinate index, \param{ycoordindx} is the second. The default values
are 1 for \param{xcoordindx} and 2 for \param{ycoordindx}, representing barrier in xy plane.
\item Symmetry barrier

\begin{record}[0.9\textwidth]
  \recentry{\entKeywordInst{symmetrybarrier}}{\componentNum\ \field{origin}{ra}}
  \recentry{}{\field{normals}{ra} \field{activemask}{ia}}
\end{record}
Implementation of symmetry barier, that allows to specify up to three planes (orthogonal ones) of symmetry.
This barrier allows to  model the symmetry of the averaged field on the boundary without
the need of modeling the other part of structure across the plane of symmetry.
It is based on modifying the integration weights of source points to take into account
the symmetry.
The potential symmetry planes are determined by specifying orthogonal right-handed coordinate system,
where axes represent the normals of corresponding symmetry planes.
Parameter \param{origin} determines the origin of the coordinate system, the \param{normals} array
contains three components of x-axis direction vector,
followed by three components of y-axis direction vector (expressed in global coordinate system).
The z-axis is determined from the orthogonality conditions.
Parameter \param{activemask} allows to specify active symmetry planes; i-th nonzero value activates the symmetry barrier for plane
with normal determined by corresponding coordinate axis (x=1, y=2, z=3).
\end{itemize}

\subsection{Load and boundary conditions}
\label{_LoadBoundaryInitialConditions}
These records specify description of  boundary conditions. The general format is
following:

\noindent
\begin{record}
  \recentry{\entKeyword{EntType}}{\componentNum}
  \recentry{}{\field{loadTimeFunction}{in}}
  \recentry{}{\optField{valType}{in} \optField{defaultDofs}{ia}}
  \recentry{}{\optField{is\-Impo\-sed\-Ti\-me\-Fun\-ction}{in}}
\end{record}

The order of particular  records is optional, boundary condition number is determined by \componentNum\ parameter.
The numbering should start from one and should end at n, where n is the number of records.
Time function value (given by \param{loadTimeFunction} parameter) is a multiplier, using which
each  component (value of loading or value of boundary condition)
describes its time variation.
The optional parameter \param{valType} allows to determine the
physical meaning of bc value, which is sometimes requared. Supported
values are (1 - temperature, 2 - force/traction,
3 - pressure, 4 - humudity, 5 - velocity, 6 - displacement).
Another optional parameter \param{defaultDofs} is used to determine which dofs the boundary condition should act
upon when new nodes are created on the boundaries.

The nonzero value of \param{isImposedTimeFunction} time function
indicates that given boundary condition is active, zero value
indicates not active boundary condition in given time (the bc does not
exist). By default, the boundary condition applies at any time.


Currently, EntType keyword can be one from
\begin{itemize}
\item Dirichlet boundary condition

\noindent
\begin{record}[0.9\textwidth]
  \recentry{\entKeywordInst{BoundaryCondition}}{\field{prescribedvalue}{rn}}
  \recentry{}{\optField{d}{rn}}
\end{record}

Represents boundary condition.
Prescribed value is specified using \param{prescribedvalue} parameter.
The physical meaning of value is fully determined by corresponding DOF.
Optionally, the prescribed value can be specified using \param{d}
parameter. It is introduced for compatibility reasons. If
\param{prescribedvalue} is specified, then \param{d} is ignored.

\item Prescribed gradient boundary condition (Dirichlet type)

\begin{record}[0.9\textwidth]
  \recentry{\entKeywordInst{PrescribedGradient}}{\field{gradient}{rm}}
  \recentry{}{\optField{cCoords}{ra}}
\end{record}

Prescribes $ v_i = d_{ij}(x_j-\bar{x}_j) $ or $ s = d_{1j}(x_j - \bar{x}_j) $
where $ v_i $ are primary unknowns, $x_j$ is the coordinate of the node, $\bar x$ is \param{cCoords} and $d$ is \param{gradient}.
The parameter \param{cCoords} defaults to zero.
This is typical boundary condition in multiscale analysis where $ d = \partial_x s$
would a macroscopic gradient at the integration point, i.e. this is a boundary condition for prolongation.
It is also convenient to use when one wants to test a arbitrary specimen for shear. 

\item Mixed prescribed gradient / pressure boundary condition (Active type)

\begin{record}[0.9\textwidth]
  \recentry{\entKeywordInst{MixedGradientPressure*}}{\field{devGradient}{ra}}
  \recentry{}{\field{pressure}{rn}}
  \recentry{}{\optField{cCoord}{ra}}
\end{record}

All boundary conditions of ensures that the deviatoric gradient and pressure is at least weakly fullfilled on the prescribed domain. They are used for computational homogenization of incompressible flow or elasticity problems.

\item Mixed prescribed gradient / pressure boundary condition (Weakly periodic type)

\begin{record}[0.9\textwidth]
  \recentry{\entKeywordInst{MixedGradientPressureWeaklyPeriodic}}{\field{order}{rn}}
\end{record}

Prescribes a periodic constant (unknown) stress tensor along the specified boundaries. For \param{order} set to 1, one obtains the same results as the Neumann boundary condition.


\item Mixed prescribed gradient / pressure boundary condition (Neumann type)

\begin{record}[0.9\textwidth]
  \recentry{\entKeywordInst{MixedGradientPressureNeumann}}{}
\end{record}

Prescribes a constant (unknown) deviatoric stress tensor along the specified boundaries.
Additional unknowns appears, $\tens{\sigma}_\dev$, which is handled by the boundary condition itself (no control from the input file).
The input devGradient is weakly fulfilled (homogenized over the elementsides). As with the the Dirichlet type, the volumetric gradient is free.
This is useful in multiscale computations of RVE's that experience incompressible behavior, typically fluid problems. In that case, the element sides should cover the entire RVE boundary.
It is also convenient to use when one wants to test a arbitrary specimen for shear, with a free volumetric part (in which case the pressure is set to zero).
Symmetry is not assumed, so rigid body rotations are removed, but translations need to be prescribed separately.

\item Mixed prescribed gradient / pressure boundary condition (Dirichlet type)

\begin{record}[0.9\textwidth]
  \recentry{\entKeywordInst{MixedGradientPressureDirichlet}}{}
\end{record}

Prescribes $ v_i = d_{\dev,ij}(x_j-\bar{x}_j) + d_\vol(x_i-\bar{x}_i)$, and a pressure $p$.
where $ v_i $ are primary unknowns, $x_j$ is the coordinate of the node, $\bar x$ is \param{cCoords} and $d_\dev$ is \param{devGradient}.
The parameter \param{cCoords} defaults to zero.
An additional unknown appears, $d_\vol$, which is handled by the boundary condition itself (no control from the input file).
This unknown is in a way related to the applied pressure.
This is useful in multiscale computations of RVE's that experience incompressible behavior, typically fluid problems. 
It is also convenient to use when one wants to test a arbitrary specimen for shear, with a free volumetric part (in which case the pressure is set to zero).



\item Nodal fluxes (loads)

\noindent
\begin{record}[0.9\textwidth]
  \recentry{\entKeywordInst{NodalLoad}}{\field{components}{ra}} \recentry{}{\optField{cstype}{in}}
\end{record}

Concentrated nodal load. The components of nodal load vector
are given by \param{components} parameter. The size of this vector
corresponds to a total number of nodal DOFs, and i-th value
corresponds to i-th DOF in associated dof manager. The load can be defined in global coordinate system (\param{cstype} =
0) or in entity - specific local coordinate system
(\param{cstype} = 1, default).
\item \mbox{}

\noindent
\begin{record}[0.9\textwidth]
  \recentry{\entKeywordInst{PrescribedTractionPressureBC}}{}
\end{record}

Represents pressure boundary condition (of Dirichlet type) due to
prescribed tractions.
In CBS algorithm formulation the prescribed traction
boundary condition leads indirectly to pressure boundary condition in
corresponding nodes. This boundary condition implements this pressure
bc. The value of bc is determined from applied tractions, that  should be specified on element edges/surfaces using
suitable boundary loads.

\item Linear constraint boundary condition

\noindent
\begin{record}[0.9\textwidth]
  \recentry{\entKeywordInst{LinearConstraintBC}}{\field{weights}{ra}} 
  \recentry{}{\optField{weightsLtf}{ia}}
  \recentry{}{\field{dofmans}{in}} \recentry{}{\field{dofs}{in}} 
  \recentry{}{\field{rhs}{rn}} \recentry{}{\optField{rhsLtf}{in}}
  \recentry{}{\field{lhstype}{ia}} \recentry{}{\field{rhsType}{ia}}
\end{record}

This boundary condition implements a linear constraint in the form $\sum_i w_ir_i = c$,
where $r_i$ are unknowns related to DOFs determined by \param{dofmans} and \param{dofs}, the weights are determined by \param{weights} and \param{weightsLtf}. The constant is determined by \param{rhs} and \param{rhsLtf} parameters. This boundary condition is introduced as additional stationary condition using Lagrange multiplier, which is an additional degree of freedom introduced by this boundary condition.

The individual DOFs are determined using dof manager numbers (\param{dofmans} array) and corresponding DOF indices (\param{dofs}). The weights corresponding to participating DOFs are specified using \param{weights} array. The weights are multiplied by value returned by load time function, associated to individual weight using optional \param{weightsLtf} array. By default, all weights are set to 1. The constant $c$ is determined by \param{rhs} parameter and it is multiplied by the value of load time function, specified using \param{rhsLtf} parameter, or by 1 by default. The characteristic component, to which this boundary condition contributes must be identified using \param{lhstype} and \param{rhsType} parameters, values of which are corresponding to CharType enum. The left hand side contribution is assembled into terms identified by \param{lhstype}. The rhs contribution is assembled into the term identified by \param{rhsType} parameter. Note, that multiple values are allowed, this allows to select all variants of stifness matrix, for example. Note, that the size of \param{dofmans}, \param{dofs}, \param{weights}, \param{weightsLtf} arrays should be equal.

\end{itemize}
\paragraph{Body loads}
\begin{itemize}
\item Volume flux (load)

\noindent
\begin{record}[0.9\textwidth]
  \recentry{\entKeywordInst{DeadWeight}}{\field{Components}{ra}}
\end{record}
Represents dead weight loading applied on element volume (for
structural elements). For transport problems, it represents the
internal source, i.e. the rate of (heat) generated per unit volume.
The magnitude of load for specific i-th DOF is computed as product of material density,
corresponding volume and i-th member of \param{Components} array.
\item Structural temperature load

\noindent
\begin{record}[0.9\textwidth]
  \recentry{\entKeywordInst{StructTemperatureLoad}~}{\field{Components}{ra}}
\end{record}

Represents temperature loading imposed to some elements. The members of
\param{Components} array represent the change of temperature (or change
of temperature gradient) corresponding to
specific element strain components. See element library manual for details.

\item Structural eigenstrain load

\noindent
\begin{record}[0.9\textwidth]
  \recentry{\entKeywordInst{StructEigenstrainLoad}~}{\field{Components}{ra}}
\end{record}
Prescribes eigenstrain (or stress-free strain) to a structural element. The array of \param{Components} is defined in the global coordinate system. The number of components corresponds to a material mode, e.g. plane stress has three components and 3D six. Periodic boundary conditions can be imposed using eigenstrains and master-slave nodes. Consider decomposition of strain into average and fluctuating part
\begin{equation}
\tens{\strain}(\vect{x}) = \langle \tens{\strain} \rangle + \tens{\strain}^*(\vect{x})
\end{equation}
where $\langle \tens{\strain} \rangle$ can be imposed as eigenstrain over the domain and the solution gives the fluctuating part $\tens{\strain}^*(\vect{x})$. Master-slave nodes have to interconnect opposing boundary nodes of a unit cell.


\end{itemize}
\paragraph{Boundary loads}
\begin{itemize}
\item Constant edge fluxes (load)

\begin{record}[0.9\textwidth]
  \recentry{\entKeywordInst{ConstantEdgeLoad}}{\field{ndofs}{in}}
  \recentry{}{\field{loadType}{in}}
  \recentry{}{\field{Components}{ra}}
  \recentry{}{\optField{dofexcludemask}{ia}}
  \recentry{}{\optField{csType}{in}}
  \recentry{}{\optField{properties}{dc}}
  \recentry{}{\optField{propertytf}{dc}}
\end{record}
\item Constant surface fluxes (load)

\begin{record}[0.9\textwidth]
  \recentry{\entKeywordInst{ConstantSurfaceLoad}}{\field{ndofs}{in}}
  \recentry{}{\field{loadType}{in}}
  \recentry{}{\field{Components}{ra}}
  \recentry{}{\optField{dofexcludemask}{ia}}
  \recentry{}{\optField{csType}{in}}
  \recentry{}{\optField{properties}{dc}}
  \recentry{}{\optField{propertytf}{dc}}
\end{record}
Represent constant edge/surface loads or boundary conditions. The \param{ndofs} parameter must be set
equal to element's number of unknowns. Parameter \param{loadType}
distinguishes the type of boundary condition. Supported values are:
2 - prescribed flux input (Neumann boundary condition), 3 - uniform distributed load or the convection (Newton) bc. The transfer coefficient is defined by keyword \param{properties}, see later.
If the boundary condition corresponds to distributed force load, the
\param{Components} array contains components of distributed load
corresponding to element unknowns.
The load is specified for all DOFs of object to which is associated.
For some types of boundary conditions the zero value of load does not mean
that the load is not applied (Newton's type of bc, for example). Then
some mask, which allows to exclude specific dofs is necessary.
The \param{dofexcludemask} parameter is introduced to alow this.
It should have the same size as \param{Components} array, and by
default is filled with zeroes. If some value of dofExcludeMask is set
to nonzero, then the corresponding componentArray
is set to zero and load is not applied for this DOF.
If the boundary condition corresponds to prescribed flux input, then
the \param{Components} array contains the components of prescribed
input flux corresponding to element unknowns.
If the boundary condition is a convection boundary condition
(\param{loadType} = 6) then \param{Components} array contains the
environmental values (temperature of the environment) corresponding to
element unknowns, and \param{properties} dictionary should contain
value of transfer (convection) coefficient (assumed to be a constant) under the key 'a'.

The properties can vary in time. Each property can have associated time function which determines its time variation. The time functions are set up using optional \param{propertytf} dictionary, containing for selected properties the corresponding time function number. The time function must be registered under the same key as in \param{properties} dictionary. The property value is then computed by product of property value (determined by \param{properties}) and corresponding time finction evaluated at given time. If no time function provided for particula property, a unit constant function is assumed.

The load can be defined in global coordinate system (\param{csType} =
0, default) or in entity - specific local coordinate system
(\param{csType} = 1).
\item Linear edge flux (load)

\begin{record}[0.9\textwidth]
  \recentry{\entKeywordInst{LinearEdgeLoad}}{\field{ndofs}{in}}
  \recentry{}{\field{loadType}{in}}
  \recentry{}{\field{Components}{ra}}
  \recentry{}{\optField{dofexcludemask}{ia}}
  \recentry{}{\optField{csType}{in}}
\end{record}

Represents linear edge load. The meanings of parameters \param{ndofs},
\param{csType}, and \param{loadType} are the same as for
\entKeywordInst{ConstantEdgeLoad}. In \param{Components}
array are stored load components for corresponding unknowns at the
beginning of edge (\param{ndofs} values), followed by values valid for
end of edge (\param{ndofs} values). The load can be defined in global coordinate system (\param{csType} =
0, default) or in entity - specific local coordinate system
(\param{csType} = 1).
\end{itemize}


\subsection{Initial conditions}
\label{_InitialConditions}
These records specify description of initial conditions. The general format is
following:

\noindent
\begin{record}
  \recentry{\entKeywordInst{InitialCondition}}{\componentNum}
  \recentry{}{\field{conditions}{dc}}
\end{record}
The order of particular  records is optional, load, boundary or initial condition number is determined by \componentNum\ parameter.
The numbering should start from one and should end at n, where n is the number of records.
Initial parameters are listed in \param{conditions} dictionary using keys followed by their initial values.
Now 'v' key represents velocity and 'a' key represents acceleration.



\subsection{Time functions records}
\label{_TimeFunctionsRecords}
These records specify description of time functions, which generally describe
time variation of components during solution. The  general format is
following:

\noindent
\begin{record}
  \recentry{\entKeyword{TimeFunctType}}{\componentNum}
  \recentry{}{\optField{initialValue}{rn}}
\end{record}

The order of these records is optional, time function number is determined by \componentNum\ parameter.
The \param{initialValue} parameter allows to control the way, how increment of receiver is evaluated for the first solution step.
This first solution step increment is evaluated as the difference of value of receiver at this first step and given initial value (which is by default set to zero).
The increments of receiver in subsequent steps are computed as a difference between receiver evaluated at given solution step and in previous step.

The numbering should start from one and should end at n, where n is the number of records.

Currently, TimeFunctType keyword can be one from
\begin{itemize}
\item Constant function

\noindent
\begin{record}[0.9\textwidth]
  \recentry{\entKeywordInst{ConstantFunction}}{\field{f(t)}{rn}}
\end{record}

Represents the constant time function, with value \param{f(t)}.

\item Peak function

\begin{record}[0.9\textwidth]
  \recentry{\entKeywordInst{PeakFunction}}{\field{t}{rn}}
  \recentry{}{\field{f(t)}{rn}}
\end{record}

Represents peak time function. If time is equal to \param{t}
value, then the value of time function is given by \param{f(t)} value,
otherwise zero value is returned.

\item Piecewise function

\begin{record}[0.9\textwidth]
  \recentry{\entKeywordInst{PiecewiseLinFunction}}{[\field{nPoints}{in} {\field{t}{ra} \field{f(t)}{ra}}] [ \field{datafile}{"string"}]}
\end{record}

Represents the piecewise time function. The particular time values in
\param{t} array should be sorted according to time scale. Corresponding time
function values are in \param{f(t)} array. Value for time, which
is not present in \param{t} is computed using liner interpolation scheme.
Number of time-value pairs is in \param{nPoints} parameter.

The second alternative allows reading input data from an external ASCII file. A hash commented line (\#) is 
skipped during reading. File name should be eclosed with " ".


\item Heaviside-like time function

\begin{record}[0.9\textwidth]
  \recentry{\entKeywordInst{HeavisideLTF}}{\field{origin}{rn}}
  \recentry{}{\field{value}{rn}}
\end{record}

Up to time, given by
parameter \param{origin}, the value of time function is zero. If time
greater than \param{origin} parameter, the value is
equal to parameter \param{value} value.
\item User defined

\begin{record}[0.9\textwidth]
  \recentry{\entKeywordInst{UsrDefLTF}}{\field{f(t)}{expr}}
  \recentry{}{\optField{dfdt(t)}{expr}}
  \recentry{}{\optField{d2fdt2(t)}{expr}}
\end{record}

Represents user defined time function.
The expressions can depend on ``t'' parameter, for which actual time will be substituted and
expression evaluated. The function is defined using \param{f(t)} parameter, and optionally, its first and second time derivatives using \param{dfdt(t)} and \param{d2fdt2(t)} parameters.
The first and second derivatives may be required, this depend on type of analysis.

Very general, but relatively slow.
\end{itemize}

\subsection{Xfem manager record and associated records}
\label{_XFEMManagerRecords}
This record specifies the number of enrichment items and simulation options
common for all enrichment items. Functions used for enrichment (e.g. Heaviside,
abs or branch functions) are not specified here, they are specified for each
enrichment item separately. The same holds for the geometrical representation of
each enrichment item (e.g. a polygon line or a circle). Currently, OOFEM
supports XFEM simulations of cracks and material interfaces in 2D. The input
format for the XFEM manager is:\\
\begin{record}
  \recentry{\entKeywordInst{XfemManager}}{\field{numberofenrichmentitems}{in}}
  \recentry{}{\field{numberofgppertri}{in}}
  \recentry{}{\field{debugvtk}{in}}
  \recentry{}{\field{vtkexport}{in}}
  \recentry{}{\field{exportfields}{in}}
\end{record}
where \param{numberofenrichmentitems} represents number of enrichment items,
\param{numberofgppertri} denotes the number of Gauss points in each subtriangle
of a cut element (default 12) and \param{debugvtk} controls if additional debug
vtk files should be written (1 activates the option, 0 is default).

The specification of an enrichment item may consist of several lines, see e.g.
the test \textit{sm/xFemCrackValBranch.in}. First, the enrichment item type is
specified together with some optional parameters according to\\
\noindent
\begin{record}
  \recentry{\entKeyword{EntType}}{\componentNum}
  \recentry{}{\field{enrichmentfront}{in}}
  \recentry{}{\field{propagationlaw}{in}}
\end{record}
where \param{enrichmentfront} specifies an enrichment front (we may for example
employ branch functions at a crack tip and Heaviside enrichment along the rest
of the crack, hence the ``front'' of the enrichment is treated separately) and
\param{propagationlaw} specifies a rule for crack propagation (this feature is
still highly experimental though). Specification of an \param{enrichmentfront}
and a \param{propagationlaw} is optional.

The next line specifies the enrichment function to be used:\\
\noindent
\begin{record}
  \recentry{\entKeyword{EntType}}{\componentNum}
\end{record}

This is followed by a line specifying the geometric description (e.g. a polygon
line or a circle) according to\\
\noindent
\begin{record}
  \recentry{\entKeyword{EntType}}{\componentNum \ extra attributes}
\end{record}
where the number and type of extra attributes to specify will vary depending on the
geometry chosen, e.g. center and radius for a circle or a number of points for a
polygon line.

If an enrichment front was specified previously, the type and properties of the
enrichment front are specified on the next line according to\\
\noindent
\begin{record}
  \recentry{\entKeyword{EntType}}{\componentNum \ extra attributes}
\end{record}

If a propagation law was specified previously, it's type and properties are also
specified on a separate line according to\\
\noindent
\begin{record}
  \recentry{\entKeyword{EntType}}{\componentNum \ extra attributes}
\end{record}


\section{Examples}
\subsection {Beam structure}
The example of input file for simple beam structure is presented.
Structure geometry and its constitutive and geometrical properties are
shown in fig. (\ref{ex01}). The linear static analysis is required,
the influence of shear is neglected. Please note that one input line
which is too long to fit to page, is separated into two lines using
backslash '$\backslash$' character, but OOFEM requires that it must be typed as one single line.
\begin{figure}[htb]
\begin{htmlonly}
  \centerline{\includegraphics[width=0.7\textwidth]{ex01}}
\end{htmlonly}
%begin{latexonly}
\centerline{\includegraphics[width=0.7\textwidth]{ex01}}
%end{latexonly}
\caption{Example 1}
\label{ex01}

\end{figure}
{\small\begin{verbatim}
test41.out
Simple Beam Structure - linear analysis
#only momentum influence to the displacements is taken into account
#beamShearCoeff is artificially enlarged.
LinearStatic 1 nsteps 3
domain 2dBeam
OutputManager tstep_all dofman_all element_all
ndofman 6 nelem 5 ncrosssect 1 nmat 1 nbc 5 nic 0 nltf 3
node 1 coords 3 0.  0.  0.  bc 3 0 1 0
node 2 coords 3 2.4 0.  0.  bc 3 0 0 0
node 3 coords 3 3.8 0.  0.  bc 3 0 0 1
node 4 coords 3 5.8 0.  1.5 bc 3 0 0 0 load 1 4
node 5 coords 3 7.8 0.  3.0 bc 3 0 1 0
node 6 coords 3 2.4 0.  3.0 bc 3 1 1 2
Beam2d 1 nodes 2 1 2 mat 1 crossSect 1 boundaryLoads  2 3 1 bodyLoads 1 5
Beam2d 2 nodes 2 2 3 mat 1 crossSect 1 DofsToCondense 1 6   bodyLoads 1 5
Beam2d 3 nodes 2 3 4 mat 1 crossSect 1 DofsToCondense 1 3
Beam2d 4 nodes 2 4 5 mat 1 crossSect 1
Beam2d 5 nodes 2 6 2 mat 1 crossSect 1 DofsToCondense 1 6
SimpleCS 1 area 0.162 Iy 0.0039366 beamShearCoeff 1.e18 thick 0.54
IsoLE 1 d 1. E 30.e6 n 0.2 tAlpha 1.2e-5
BoundaryCondition  1 loadTimeFunction 1 prescribedvalue 0.0
BoundaryCondition  2 loadTimeFunction 2 prescribedvalue -0.006e-3
ConstantEdgeLoad 3 loadTimeFunction 1 Components 3 0. 10. 0.0\
                   loadType 3 ndofs 3
NodalLoad 4 loadTimeFunction 1 Components 3 -18.0 24.0 0.0
StructTemperatureLoad 5 loadTimeFunction 3 Components 2 30.0 -20.0
PeakFunction 1 t 1.0 f(t) 1.
PeakFunction 2 t 2.0 f(t) 1.
PeakFunction 3 t 3.0 f(t) 1.
\end{verbatim}}
\subsection {Plane stress example}
\begin{figure}[htb]
\begin{htmlonly}
  \centerline{\includegraphics[width=0.7\textwidth]{ex02}}
\end{htmlonly}
%begin{latexonly}
\centerline{\includegraphics[width=0.7\textwidth]{ex02}}
%end{latexonly}
\caption{Example 2}
\label{ex02}
\end{figure}
{\small\begin{verbatim}
patch100.out
Patch test of PlaneStress2d elements -> pure compression
LinearStatic nsteps 1
domain 2dPlaneStress
OutputManager tstep_all dofman_all element_all
ndofman 8 nelem 5 ncrosssect 1 nmat 1 nbc 2 nic 0 nltf 1
node 1 coords 3  0.0   0.0   0.0  bc 2 1 1
node 2 coords 3  0.0   4.0   0.0  bc 2 1 1
node 3 coords 3  2.0   2.0   0.0  bc 2 0 0
node 4 coords 3  3.0   1.0   0.0  bc 2 0 0
node 5 coords 3  8.0   0.8   0.0  bc 2 0 0
node 6 coords 3  7.0   3.0   0.0  bc 2 0 0
node 7 coords 3  9.0   0.0   0.0  bc 2 0 1  load 1 2
node 8 coords 3  9.0   4.0   0.0  bc 2 0 1  load 1 2
PlaneStress2d 1 nodes 4 1 4 3 2  crossSect 1 mat 1 NIP 1
PlaneStress2d 2 nodes 4 1 7 5 4  crossSect 1 mat 1 NIP 1
PlaneStress2d 3 nodes 4 4 5 6 3  crossSect 1 mat 1 NIP 1
PlaneStress2d 4 nodes 4 3 6 8 2  crossSect 1 mat 1 NIP 1
PlaneStress2d 5 nodes 4 5 7 8 6  crossSect 1 mat 1 NIP 1
SimpleCS 1 thick 1.0 width 1.0
IsoLE 1 d 0. E 15.0 n 0.25 talpha 1.0
BoundaryCondition  1 loadTimeFunction 1 prescribedvalue 0.0
NodalLoad 2 loadTimeFunction 1 Components 2 2.5 0.0
ConstantFunction 1 f(t) 1.0
\end{verbatim}}
%\vspace*{1.5cm}
%\begin{flushright}
%	 Printed on \today
%\end{flushright}


%\vfill
\section{Examples - parallel mode}
\subsection{Node-cut example}
The example shows explicit direct integration analysis of
simple structure with two DOFs.
The geometry and partitioning is sketched in
fig. \ref{nodecut-ex01}.

\begin{figure}[htb]
\begin{htmlonly}
  \centerline{\includegraphics[width=0.7\textwidth]{poofem_ex01}}
\end{htmlonly}
%begin{latexonly}
\centerline{\includegraphics[width=0.7\textwidth]{poofem_ex01}}
%end{latexonly}
\caption{Node-cut partitioning example: (a) whole geometry,
(b) partition~0, (c) partition~1.}
\label{nodecut-ex01}
\end{figure}

\begin{verbatim}
#
# partition 0
#
partest.out.0
Parallel test of explicit oofem computation
#
NlDEIDynamic nsteps 3 dumpcoef 0.0 deltaT 1.0 NodeCutMode
domain 2dTruss
#
OutputManager tstep_all dofman_all element_all
ndofman 2 nelem 1 ncrosssect 1 nmat 1 nbc 2 nic 0 nltf 1
#
Node 1 coords 3 0. 0. 0. bc 2 1 1
Node 2 coords 3 0. 0. 2. bc 2 1 0 Shared partitions 1 1
Truss2d 1 nodes 2 1 2 mat 1 crossSect 1
SimpleCS 1 thick 0.1 width 10.0
IsoLE 1  tAlpha 0.000012  d 10.0  E 1.0  n 0.2
BoundaryCondition  1 loadTimeFunction 1 conditions 1 d 0.0
NodalLoad 2 loadTimeFunction 1 Components 2 0. 1.0
ConstantFunction 1 f(t) 1.0

#
# partition 1
#
partest.out.1
Parallel test of explicit oofem computation
#
NlDEIDynamic nsteps 3 dumpcoef 0.0 deltaT 1.0 NodeCutMode
domain 2dTruss
#
OutputManager tstep_all dofman_all element_all
ndofman 2 nelem 1 ncrosssect 1 nmat 1 nbc 2 nic 0 nltf 1
#
Node 2 coords 3 0. 0. 2. bc 2 1 0 Shared partitions 1 0
Node 3 coords 3 0. 0. 4. bc 2 1 0 load 1 2
Truss2d 2 nodes 2 2 3 mat 1 crossSect 1
SimpleCS 1 thick 0.1 width 10.0
IsoLE 1  tAlpha 0.000012  d 10.0  E 1.0  n 0.2
BoundaryCondition  1 loadTimeFunction 1 conditions 1 d 0.0
NodalLoad 2 loadTimeFunction 1 Components 2 0. 1.0
ConstantFunction 1 f(t) 1.0
\end{verbatim}

\subsection{Element-cut example}
The example shows explicit direct integration analysis of
simple structure with two DOFs. The geometry and partitioning is sketched in
fig. \ref{nodecut-ex01}.

\begin{figure}[htb]
\begin{htmlonly}
  \centerline{\includegraphics[width=0.7\textwidth]{poofem_ex02}}
\end{htmlonly}
%begin{latexonly}
\centerline{\includegraphics[width=0.7\textwidth]{poofem_ex02}}
%end{latexonly}
\caption{Element-cut partitioning example: (a) whole geometry,
(b) partition~0, (c) partition~1.}
\label{elmentcut-ex02}
\end{figure}

\begin{verbatim}
#
# partition 0
#
partest2.out.0
Parallel test of explicit oofem computation
#
NlDEIDynamic nsteps 5 dumpcoef 0.0 deltaT 1.0 ElementCutMode
domain 2dTruss
#
OutputManager tstep_all dofman_all element_all
ndofman 3 nelem 2 ncrosssect 1 nmat 1 nbc 2 nic 0 nltf 1
#
Node 1 coords 3 0. 0. 0. bc 2 1 1
Node 2 coords 3 0. 0. 2. bc 2 1 0
Node 3 coords 3 0. 0. 4. bc 2 1 0 \
       Remote partitions 1 1 load 1 2
Truss2d 1 nodes 2 1 2 mat 1 crossSect 1
Truss2d 2 nodes 2 2 3 mat 1 crossSect 1
SimpleCS 1 thick 0.1 width 10.0
IsoLE 1  tAlpha 0.000012  d 10.0  E 1.0  n 0.2
BoundaryCondition  1 loadTimeFunction 1 conditions 1 d 0.0
NodalLoad 2 loadTimeFunction 1 Components 2 0. 1.0
ConstantFunction 1 f(t) 1.0


#
# partition 1
#
partest2.out.1
Parallel test of explicit oofem computation
#
NlDEIDynamic nsteps 5 dumpcoef 0.0 deltaT 1.0 ElementCutMode
domain 2dTruss
#
OutputManager tstep_all dofman_all element_all
ndofman 2 nelem 1 ncrosssect 1 nmat 1 nbc 2 nic 0 nltf 1
#
Node 2 coords 3 0. 0. 2. bc 2 1 0 Remote partitions 1 0
Node 3 coords 3 0. 0. 4. bc 2 1 0 load 1 2
Truss2d 2 nodes 2 2 3 mat 1 crossSect 1
SimpleCS 1 thick 0.1 width 10.0
IsoLE 1  tAlpha 0.000012  d 10.0  E 1.0  n 0.2
BoundaryCondition  1 loadTimeFunction 1 conditions 1 d 0.0
NodalLoad 2 loadTimeFunction 1 Components 2 0. 1.0
ConstantFunction 1 f(t) 1.0
\end{verbatim}

\section{Figures}
%\subsection {Node cut mode}

\begin{figure}[htb]
\begin{htmlonly}
  \centerline{\includegraphics[width=0.7\textwidth]{nodecut0cb}}
\end{htmlonly}
%begin{latexonly}
\centerline{\includegraphics[width=0.7\textwidth]{nodecut0cb}}
%end{latexonly}
\caption{Node-cut partitioning.}
\label{nodecut}
\end{figure}

\begin{figure}[htb]
\begin{htmlonly}
  \centerline{\includegraphics[width=0.7\textwidth]{nodecut1cb}}
\end{htmlonly}
%begin{latexonly}
\centerline{\includegraphics[width=0.7\textwidth]{nodecut1cb}}
%end{latexonly}
\caption{Node-cut partitioning - local constitutive mode.}
\label{nodecut-lm}
\end{figure}

\begin{figure}[htb]
\begin{htmlonly}
  \centerline{\includegraphics[width=0.7\textwidth]{nodecutnonloc1}}
\end{htmlonly}
%begin{latexonly}
\centerline{\includegraphics[width=0.7\textwidth]{nodecutnonloc1}}
%end{latexonly}
\caption{Node-cut partitioning - nonlocal constitutive mode.}
\label{nodecut-nlm}
\end{figure}

%\subsection {Element cut mode}

\begin{figure}[htb]
\begin{htmlonly}
  \centerline{\includegraphics[width=0.7\textwidth]{elementcut0}}
\end{htmlonly}
%begin{latexonly}
\centerline{\includegraphics[width=0.7\textwidth]{elementcut0}}
%end{latexonly}
\caption{Element-cut partitioning.}
\label{elmentcut}
\end{figure}

\begin{figure}[htb]
\begin{htmlonly}
  \centerline{\includegraphics[width=0.7\textwidth]{elementcut1}}
\end{htmlonly}
%begin{latexonly}
\centerline{\includegraphics[width=0.7\textwidth]{elementcut1}}
%end{latexonly}
\caption{Element-cut partitioning, local constitutive mode.}
\label{elmentcut-lm}
\end{figure}

\end{document}



